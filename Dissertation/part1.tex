\chapter{Структура порождающей модели матрицы кросс-спектральной плотноти. 
         Подпространство протечки сигнала.} \label{chapt1}

\section{Постановка задачи оценки фазовой синхронности по неинвазивным измерениям} \label{sect1_1}

Мы можем сделать \textbf{жирный текст} и \textit{курсив}.

Рассмотрим типичную постановку эксперимента по изучению некоторых
нейрофизиологических явлений с помощью ЭЭГ/МЭГ.
Предположим, что было сделано К записей результатов тестов гомогенной электрофизической
активности объекта $\alpha$ продолжительностью $\delta t$ с помощью устройства, имеющего m сенсоров.
Более того, допустим, что информация об анатомии мозга объекта α известна (этого можно достичь,
используя стандартную модель мозга или снимок МРТ мозга объекта).
Последнее допущение позволяет нам ввести пространство источников, представляющее собой сетку,
состоящую из n точек, аппроксимирующую поверхность мозга объекта (см.рис.1).
Решая так называемую прямую задачу [2], мы можем осуществить отображение точек пространства
источников n в m-мерное пространство сенсоров
(обычно n >> m, так как для качественной аппроксимации нам нужно намного
больше точек поверхности мозга, чем число сенсоров, которым мы располагаем)
с оператором отображения G.
Оператор G решает уравнения Максвелла для точечных источников электромагнитной активности,
расположенных в каждом узле сетки, определяющей конфигурацию источников в пространстве,
и пока уравнения Максвелла остаются линейными, оператор G также является линейным [8].
Мы также должны учесть влияние шума источники которого располагаются во внешней среде
и который принимается как белый шум.
Теперь мы можем записать порождающую модель измерений сенсорами

\begin{equation}
	\mathbf{x}_k(t) = \mathbf{G} \cdot \mathbf{s}_k(t) + \mathbf{\omega}_k(t),
	\label{gm_ts}
\end{equation}
где $\mathbf{s}_k(t)$  вектор-столбец активаций источников размера $n \times 1$
$\mathbf{x}_k(t)$ вектор-столбец сигналов, снятых сенсорами,
размера $m \times 1$, $t$ --- время, $\mathbf{G}$ --- $m \times n$ матрица линейного отображения пространства источников в пространство сигналов.
Индекс $k$ обозначает номер эпохи.
Применим к (\ref{gm_ts}) частотное преобразование. Существует несколько способов произвести
эту операцию, например, --- вейвлет-преобразование или применение узкополосного фильтра
с задаваемыми резонансными частотами и последующим аналитическим представлением сигнала.
Мы не будем вдаваться в подробности теории частотно-временных преобразований;
для подробного математизированного изложения предмета см.
\cite{Oppenheim1998}, о  применении  частотно-временных преобращований к обработке
электрофизиологических данных можно прочитать в \cite{Freeman}.
Мы только упомянем, что вне зависимости от особенностей выбранного
метода образом частотно-временного преобразования матрицы $\mathbf{x}$ размера $m \times T$
будет комплексный тензор каждый временной срез которого содержит в себе информацию
о фазовом и амплитудном спектре сигнала в каждый момент времени.
Чтобы упростить вывод, мы зафиксируем частоту и примем,
что соотношения справедливы для всех остальных частот преобразования. В итоге мы можем записать:

2
где комплексные изображения Для упрощения результата опустим индекси ограничимся одной частотой: 

3
Теперь, если мы умножим каждой записи на их эрмитово-сопряженные матрицы и усредним результат, мы получим генеративную модель кросс-спектра уровня сенсоров:

4
где оператор вводится для усреднения по испытаниям. Пока мы считаем шум белым и что сигнал уровня сенсоров   и шум взаимно независимы, второй и третий члены уравнения (4) равны нулю в том случае, когда число испытаний достаточно велико. Также отметим, что и представляют собой внешние произведения комплексно сопряженных векторов с векторами начальных значений, и, следовательно, являются эрмитовыми матрицами, что означает, что значения, находящиеся на их диагоналях, принадлежат области вещественных чисел. Это свойство сохраняется после усреднения.
Более того, раз элементы векторов являются комплексными числами, они могут быть представлены в виде где соответствует мгновенной фазе сигнала, а амплитуде. Следовательно, или, если для пояснения мы примем, что амплитуды и фазы независимы по умолчанию (что является спорным утверждением для электрофизиологических данных [12], [14]), мы получим, что Из последнего соотношения можно сделать некоторые выводы. Во-первых, элементы матрицы кросс-спектра представляют собой степень стабильности разности фаз в испытаниях. Если разность фаз достаточно равномерно распределена по всему возможному интервалу принимаемых значений, то среднее приблизительно равно нулю. Напротив, если разность фаз сохраняется постоянной от испытания к испытанию, результирующий коэффициент будет отличен от нуля, что соответствует случаю установления коннективности между сигналами. Во-вторых, если разность фаз мала, элементы взаимного спектра могут быть близки к ненулевому вещественному числу.
Введем следующее обозначение для матрицы кросс-спектральной плотности: 

5
Используя определение (5) и опуская начальные условия, (4) перепишется в виде

6
Теперь рассмотрим подробнее главную диагональ матрицы Как мы упомянули ранее, элементы главной диагонали этой матрицы являются вещественными числами и они представляют собой значения мощностей сигналов пространства источников, имеющих частоту Структура генеративной модели показывает, что после отображения оператором из пространства источников в пространство сенсоров с помощью оператора с эти мощностные члены будут смешаны с истинной коннективностью, и так как начальная система была сильно неопределена (условие, что n >> m), разделить сигналы не представляется возможным. Это объясняет эффект объемной проводимости на математическом языке. 
Чтобы картина стала более детальной, представьте ситуацию, где между источниками не существует никаких связей, но при этом некоторые участки мозга активны. В таком случае, все элементы матрицы лежащие вне главной диагонали, будут равны нулю, но для это выполняться не будет. Пары сенсоров, расположенные близко к активным участкам мозга, будут иметь большие коэффициенты кросс-спектра, что приведет к ложному обнаружению коннективности между двумя источниками.
Ранее упомянутый метод ImCoh (в настоящее время, вероятно, наиболее часто используемый метод измерения коннективности в данной дисциплине) предполагает, что на рассмотрение берутся только мнимые части уравнения (6), что уберегает нас от негативного эффекта объемного сопротивления, но при этом мы также теряем вещественную часть полезного сигнала.
Для более полного использования вещественной части взаимного спектра мы предлагаем другой подход к устранению эффекта объемной проводимости. Расширим выражение умножения матриц 

7
где столбец матрицы называемый топографией источника р, поскольку он определяет отображение сигнала, поступающего от источника р, на сенсоре. Можно увидеть, что кросс-спектр пространства сенсоров представляет собой линейную комбинацию выходных данных топографий с коэффициентами, являющимися элементами кросс-спектра пространства источников. Векторизуем следующее уравнение:

8

2.1. Произведение Крокенера
Для упрощения записи будем использовать понятие произведения Крокенера, определяющее для матриц А и В, имеющих размеры и соответственно, матрицу которая записывается следующим образом: 

9
Произведение Крокенера является билинейной ассоциативной операцией:

10-13
Заметим, что произведение Крокенера несимметрично: Приведем другие полезные соотношения:

14,15
Существует особо интересное для нас свойство произведения Крокенера, связывающее его с процедурой векторизации. Для матриц мы получаем (доказательство см. в [10]):

16
Отметим, что в нашем случае приведенное выражение принимает самую простую форму:

17
с представляющим собой вектор-столбец размера 
Перепишем выражение (8), используя новые обозначения:

18
Теперь можно видеть, что векторизованный кросс-спектр уровня сенсоров теперь представлен линейной комбинацией векторов в векторном пространстве. Мы назовем эти векторы 2-топографиями. Мы также знаем, что эффект объемной проводимости, от которого нам нужно избавиться, появляется в 2-топографиях в специальной форме (n векторов в целом), которая известна с тех пор как мы узнаем особую форму оператора Следовательно, мы можем спроецировать выражение (18) из линейного пространства в охваченные 2-топографиями источников объемной проводимости.
2.2. Создание проецирования
Прежде чем мы создадим проектор из VC-подмножества, рассмотрим, как подпространство объемной проводимости относится к линейному диапазону 2-топографий вещественной и мнимой частей кросс-спектра генеративной модели.
Сначала выделим в уравнении (18) вещественную и мнимую части. Заметим, что пока матрица является эрмитовой, (линия сверху обозначает комплексное сопряжение):

19
Примем во внимание нижний индекс в операторах суммирования.
Для удобства обозначим линейное пространство, охваченное объемной проводимости 2-топографий, как диапазон вещественной части 2-топографии как и диапазон мнимой части – как 
Из уравнения (19) видно, что 2-топографии вещественной и мнимой частей имеют разные структуры. А именно, векторы являются симметричными, пока 2-топографии мнимых частей антисимметричны по индексам p, q. Вопрос заключается в том, как это структурное отличие влияет на пространства и Покажем, что мнимая часть 2-топографии ортогональна к векторам, охватывающих подпространство объемной проводимости: 

20
Равенство * сохраняется, так как и являются скалярными величинами, и мы можем опустить операциюи заменить множители. Из (20) можно видеть, что подпространство объемной проводимости ортогонально мнимой части подпространства. Для вещественной части такое соотношение не сохраняется:

21
После проведенных операций можно увидеть, что проекция из подпространства объемной проводимости будет влиять на вещественную часть истинной проводимости, следовательно, нужно добиться того, чтобы объемная проводимость была удалена из сигнала тогда, когда действие вещественной части кросс-спектра мало настолько, насколько это возможно (см.рис.2). Для достижения этой цели необходимо уменьшить размерность подпространства объемной проводимости неким оптимальным способом.
Принимая во внимание все вышеперечисленное, мы продолжаем создавать проектор. Рассмотрим матрицу, состоящую из векторов-столбцов, охватывающих пространство объемной проводимости, поставленных вместе горизонтально:

22
Для начала представим разложение SVD матрицей 

23
Подпространство объемной проводимости и вещественное подпространство перекрываются и являются ортогональными к подпространству мнимых компонент кросс-спектра. Пересечение предшествующих подпространств включает в себя вклад объемной проводимости и вещественной части взаимного спектра источников, расположенных близко друг к другу и имеющих активность, которую может быть выражена малой величиной усредненной разности фаз. 
В соответствие со свойствами разложения VC-пространства, взятие первых r столбцов матрицы возвращает ортонормированный базис r-мерного линейного пространства, представляющего собой наилучшим приближением n-мерного подпространства объемной проводимости. Используем эти r векторов для проектора с уменьшенной размерностью из подпространства объемной проводимости:

24,25
Итак, мы построили проектор из подпространства объемной проводимости уменьшенного ранга r.
Умножая уравнение (18) на его проекцию приводит к желаемому результату – генеративная модель для кросс-спектра пространства сенсоров избавляется от объемной проводимости, где параметр r контролирует взаимодействие между желаемым уровнем очистки от объемной проводимости и того, насколько сильно при этом затрагивается вещественная часть взаимного спектра. Теперь мы можем написать финальную формулу кросс-спектра пространства сенсоров, спроектированного из VC-подпространства:

26
Из уравнения выше видно, что предложенная проекция оставляет нас в линейной области. Элементы нового векторизованного кросс-спектра теперь могут быть использованы для исследования и анализа коннективностей в мозге на уровнях сенсоров и источников.
2.3. Расположение случайных ориентаций диполя
С точки зрения анатомии каждая топография исходной модели G представляет собой создаваемое токами электромагнитное поле, распространяющееся через апикальные дендриты кортикальных пирамидальных нейронов. Если апикальные дендриты располагаются перпендикулярно к кортикальной мантии, токи имеют то же направление. Следовательно, создание правильного прямого оператора требует точнейшей кортикальной реконструкции. Хотя современное моделирование с использованием метода МРТ позволяет получать детальную реконструкцию мозга с количеством точек от нескольких сотен до тысяч, использование большого количества полученных точек в пространстве источников приводит к значительному спаду производительности анализирующих алгоритмов вследствие задействования больших объемов памяти и времени обработки огромных массивов данных.  
Данное обстоятельство приводит к необходимости разредить сетку источников для ускорения вычислений. Эту процедуру следует выполнять осторожно, так как разрежение приводит к потере информации о направлениях ориентаций нормалей. Неопределенность исходит из того, что после разрежения узлы сетки будут представлять собой пространственные пятна на кортикальной поверхности с разными степенями кривизны. Изменение местоположения активации внутри отдельного пятна приводит к сдвигу нормали и соответствующей топографии. Общий подход к оценке данного эффекта заключается в том, чтобы выделить нормаль к каждому пятну, имеющую свободную ориентацию, с помощью введения дополнительных параметров в модель [13].
Для того чтобы это сделать, следует представить топографию в месте p в виде линейной комбинации трех ортогональных друг к другу векторов топографии, размещенных в одной точке, следующим образом:

27
В случае, если измерения проводятся с помощью МЭГ, с того момента как создаваемое диполем с радиальной ориентацией магнитное поле, находящееся вне сферического проводника, равно нулю, введенная тройка векторов может быть перемещена парой диполей в локальную касательную плоскость, рассчитываемую для каждого узла. Следовательно, уравнение МЭГ (27) перепишется в виде

28
Соответственно, мы должны изменить выражение и для топографии объемной проводимости:

29
Таким образом, мы видим, что 2-топографии объемной проводимости теперь представлены тройками векторов, а именно Результирующая 2-топография объемной проводимости зависит только от параметра  и угла между направлением топографии и ее х-компоненты и, таким образом, и результирующий вектор запишется в виде
Но это 1-параметрическое семейство не может быть приведено к линейному пространству с размерностью меньше 3, чтобы проецировать из него. Значит, мы должны добавить все 3 2-топографии источника р в матрицу F для построения проектора:
%\newpage
%============================================================================================================================

\section{Ссылки} \label{sect1_2}
Сошлёмся на библиографию.
Одна ссылка: \cite[с.~54]{Sokolov}\cite[с.~36]{Gaidaenko}.
Две ссылки: \cite{Sokolov,Gaidaenko}.
Много ссылок: %\cite[с.~54]{Lermontov,Management,Borozda} % такой «фокус» вызывает biblatex warning относительно опции sortcites, потому что неясно, к какому источнику относится уточнение о страницах, а bibtex об этой проблеме даже не предупреждает
\cite{Lermontov,Management,Borozda,Marketing,Constitution,FamilyCode,Gost.7.0.53,Razumovski,Lagkueva,Pokrovski,Sirotko,Lukina,Methodology,Encyclopedia,Nasirova,Berestova,Kriger}.
И~ещё немного ссылок:
\cite{Article,Book,Booklet,Conference,Inbook,Incollection,Manual,Mastersthesis,Misc,Phdthesis,Proceedings,Techreport,Unpublished}.
\cite{medvedev2006jelektronnye, CEAT:CEAT581, doi:10.1080/01932691.2010.513279,Gosele1999161,Li2007StressAnalysis, Shoji199895,test:eisner-sample,AB_patent_Pomerantz_1968,iofis_patent1960}

%Попытка реализовать несколько ссылок на конкретные страницы для стандартной реализации:[\citenum{Sokolov}, с.~54; \citenum{Gaidaenko}, с.~36].

%Несколько источников мультицитата (только в biblatex)
%\cites[vii--x, 5, 7]{Sokolov}[v"--~x, 25, 526]{Gaidaenko} поехали дальше

Ссылки на собственные работы:~\cite{vakbib1, confbib1}

Сошлёмся на приложения: Приложение \ref{AppendixA}, Приложение \ref{AppendixB2}.

Сошлёмся на формулу: формула \eqref{eq:equation1}.

Сошлёмся на изображение: рисунок \ref{img:knuth}.

%\newpage
%============================================================================================================================

\section{Формулы} \label{sect1_3}

Благодаря пакету \textit{icomma}, \LaTeX~одинаково хорошо воспринимает в качестве десятичного разделителя и запятую ($3,1415$), и точку ($3.1415$).

\subsection{Ненумерованные одиночные формулы} \label{subsect1_3_1}

Вот так может выглядеть формула, которую необходимо вставить в строку по тексту: $x \approx \sin x$ при $x \to 0$.

А вот так выглядит ненумерованая отдельностоящая формула c подстрочными и надстрочными индексами:
\[
(x_1+x_2)^2 = x_1^2 + 2 x_1 x_2 + x_2^2
\]

При использовании дробей формулы могут получаться очень высокие:
\[
  \frac{1}{\sqrt{2}+
  \displaystyle\frac{1}{\sqrt{2}+
  \displaystyle\frac{1}{\sqrt{2}+\cdots}}}
\]

В формулах можно использовать греческие буквы:
\[
\alpha\beta\gamma\delta\epsilon\varepsilon\zeta\eta\theta\vartheta\iota\kappa\lambda\\mu\nu\xi\pi\varpi\rho\varrho\sigma\varsigma\tau\upsilon\phi\varphi\chi\psi\omega\Gamma\Delta\Theta\Lambda\Xi\Pi\Sigma\Upsilon\Phi\Psi\Omega
\]

\def\slantfrac#1#2{ \hspace{3pt}\!^{#1}\!\!\hspace{1pt}/
  \hspace{2pt}\!\!_{#2}\!\hspace{3pt}
} %Макрос для красивых дробей в строчку (например, 1/2)
Для красивых дробей (например, в индексах) можно добавить макрос
\verb+\slantfrac+ и писать $\slantfrac{1}{2}$ вместо $1/2$.
%\newpage
%============================================================================================================================

\subsection{Ненумерованные многострочные формулы} \label{subsect1_3_2}

Вот так можно написать две формулы, не нумеруя их, чтобы знаки равно были строго друг под другом:
\begin{align}
  f_W & =  \min \left( 1, \max \left( 0, \frac{W_{soil} / W_{max}}{W_{crit}} \right)  \right), \nonumber \\
  f_T & =  \min \left( 1, \max \left( 0, \frac{T_s / T_{melt}}{T_{crit}} \right)  \right), \nonumber
\end{align}

Выровнять систему ещё и по переменной $ x $ можно, используя окружение \verb|alignedat| из пакета \verb|amsmath|. Вот так: 
\[
    |x| = \left\{
    \begin{alignedat}{2}
        &&x, \quad &\text{eсли } x\geqslant 0 \\
        &-&x, \quad & \text{eсли } x<0
    \end{alignedat}
    \right.
\]
Здесь первый амперсанд (в исходном \LaTeX\ описании формулы) означает выравнивание по~левому краю, второй "--- по~$ x $, а~третий "--- по~слову <<если>>. Команда \verb|\quad| делает большой горизонтальный пробел.

Ещё вариант:
\[
    |x|=
    \begin{cases}
    \phantom{-}x, \text{если } x \geqslant 0 \\
    -x, \text{если } x<0
    \end{cases}
\]

Кроме того, для  нумерованых формул \verb|alignedat|  делает вертикальное
выравнивание номера формулы по центру формулы. Например,  выравнивание компонент вектора:
\begin{equation}
 \label{eq:2p3}
 \begin{alignedat}{2}
{\mathbf{N}}_{o1n}^{(j)} = \,{\sin} \phi\,n\!\left(n+1\right)
         {\sin}\theta\,
         \pi_n\!\left({\cos} \theta\right)
         \frac{
               z_n^{(j)}\!\left( \rho \right)
              }{\rho}\,
           &{\boldsymbol{\hat{\mathrm e}}}_{r}\,+   \\
+\,
{\sin} \phi\,
         \tau_n\!\left({\cos} \theta\right)
         \frac{
            \left[\rho z_n^{(j)}\!\left( \rho \right)\right]^{\prime}
              }{\rho}\,
            &{\boldsymbol{\hat{\mathrm e}}}_{\theta}\,+   \\
+\,
{\cos} \phi\,
         \pi_n\!\left({\cos} \theta\right)
         \frac{
            \left[\rho z_n^{(j)}\!\left( \rho \right)\right]^{\prime}
              }{\rho}\,
            &{\boldsymbol{\hat{\mathrm e}}}_{\phi}\:.
\end{alignedat}
\end{equation}

Ещё об отступах. Иногда для лучшей <<читаемости>> формул полезно
немного исправить стандартные интервалы \LaTeX\ с учётом логической
структуры самой формулы. Например в формуле~\ref{eq:2p3} добавлен
небольшой отступ \verb+\,+ между основными сомножителями, ниже
результат применения всех вариантов отступа:
\begin{align*}
\backslash! &\quad f(x) = x^2\! +3x\! +2 \\
  \mbox{по-умолчанию} &\quad f(x) = x^2+3x+2 \\
\backslash, &\quad f(x) = x^2\, +3x\, +2 \\
\backslash{:} &\quad f(x) = x^2\: +3x\: +2 \\
\backslash; &\quad f(x) = x^2\; +3x\; +2 \\
\backslash \mbox{space} &\quad f(x) = x^2\ +3x\ +2 \\
\backslash \mbox{quad} &\quad f(x) = x^2\quad +3x\quad +2 \\
\backslash \mbox{qquad} &\quad f(x) = x^2\qquad +3x\qquad +2
\end{align*}


Можно использовать разные математические алфавиты:
\begin{align}
\mathcal{ABCDEFGHIJKLMNOPQRSTUVWXYZ} \nonumber \\
\mathfrak{ABCDEFGHIJKLMNOPQRSTUVWXYZ} \nonumber \\
\mathbb{ABCDEFGHIJKLMNOPQRSTUVWXYZ} \nonumber
\end{align}

Посмотрим на систему уравнений на примере аттрактора Лоренца:

\[ 
\left\{
  \begin{array}{rl}
    \dot x = & \sigma (y-x) \\
    \dot y = & x (r - z) - y \\
    \dot z = & xy - bz
  \end{array}
\right.
\]

А для вёрстки матриц удобно использовать многоточия:
\[ 
\left(
  \begin{array}{ccc}
  	a_{11} & \ldots & a_{1n} \\
  	\vdots & \ddots & \vdots \\
  	a_{n1} & \ldots & a_{nn} \\
  \end{array}
\right)
\]


%\newpage
%============================================================================================================================
\subsection{Нумерованные формулы} \label{subsect1_3_3}

А вот так пишется нумерованая формула:
\begin{equation}
  \label{eq:equation1}
  e = \lim_{n \to \infty} \left( 1+\frac{1}{n} \right) ^n
\end{equation}

Нумерованых формул может быть несколько:
\begin{equation}
  \label{eq:equation2}
  \lim_{n \to \infty} \sum_{k=1}^n \frac{1}{k^2} = \frac{\pi^2}{6}
\end{equation}

Впоследствии на формулы (\ref{eq:equation1}) и (\ref{eq:equation2}) можно ссылаться.

Сделать так, чтобы номер формулы стоял напротив средней строки, можно, используя окружение \verb|multlined| (пакет \verb|mathtools|) вместо \verb|multline| внутри окружения \verb|equation|. Вот так:
\begin{equation} % \tag{S} % tag - вписывает свой текст 
  \label{eq:equation3}
    \begin{multlined}
        1+ 2+3+4+5+6+7+\dots + \\ 
        + 50+51+52+53+54+55+56+57 + \dots + \\ 
        + 96+97+98+99+100=5050 
    \end{multlined}
\end{equation}

Используя команду \verb|\labelcref| из пакета \verb|cleveref|, можно
красиво ссылаться сразу на несколько формул
(\labelcref{eq:equation1,eq:equation3,eq:equation2}), даже перепутав
порядок ссылок \verb|(\labelcref{eq:equation1,eq:equation3,eq:equation2})|.

