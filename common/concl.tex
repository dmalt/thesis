%% Согласно ГОСТ Р 7.0.11-2011:
%% 5.3.3 В заключении диссертации излагают итоги выполненного исследования, рекомендации, перспективы дальнейшей разработки темы.
%% 9.2.3 В заключении автореферата диссертации излагают итоги данного исследования, рекомендации и перспективы дальнейшей разработки темы.
\begin{enumerate}
  \item Был проведен обзор исследований изменения функциональной коннективности
      мозга при различных паталогиях.
  \item Был разработан метод очистки данных ЭЭГ и МЭГ от протечки сигнала, на
      основе которого было разработано семейство алгоритмов оценки фазовой синхронности,
      позволяющих находить сети с близкими к нулю фазовыми задержками.
  \item Задача оценки фазовой синхронности в условиях протечки сигнала была
      сформулирована и решена как задача оптимальной фильтрации.
  \item Был предложен алгоритм, позволяющий обнаруживать сети с близкими
      к нулю фазовыми задержками, оптимальный в глобальном смысле и позволяющий
      справиться с проблемой ложноположительных срабатываний второго рода, вызванных
      протечкой сигнала.
  \item Было проведено численное исследование свойств предложенной
      проекции, показавшее, что разработанная методика позволяет
      подавить вклад подпространства протечки сигнала в оцененную
      на сенсорах матрицу кросс-спектральной плотности мощности.
  \item Численное исследование свойств метода проекции показало, что
      разработанная методика позволяет находить сети с близкими к нулю фазовыми задержками в условиях
      неинвазивных МЭГ измерений, которые характеризуются значительной протечкой
      сигнала между источниками.
  \item Сравнение с имеющимися на данный момент алгоритмами оценки коннективности
      по неинвазивным данным на основе симуляций показало значительное превосходство
      предложенной техники обнаружения сетей в условиях малых фазовых задержек.
  \item Применение метода очистки от протечки сигнала к реальным данным позволило
      обнаружить физиологически правдоподобные сети, которые невозможно обнаружить
      другими способами.
  \item Было проведенно численное исследование влияния значений ранга предложенной проекции
      на свойства алгоритма, которое позволило получить эвристику для выбора ранга.
  \item Численное исследование влияния неточностей прямой модели на качество решений
      предложенного алгоритма показало, что характерные для реальных записей
      диапазоны ошибок в оценке прямой модели слабо сказываются на качестве
      получаемых решений.
  \item Для выполнения поставленных задач был создан
      пакет утилит в среде MATLAB, в который входят средства генерации тестовых
      данных, визуализации пространственной и временной структуры сетей и наконец
      программные реализации разработанных и использованных для валидации алгоритмов.
  \item Наработки, полученные в ходе работы над данной диссертацией,
      были внедрены в пакеты программ Visbrain и Neuropycon, доступные для публичного
      использования.
\end{enumerate}
