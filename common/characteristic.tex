
% Обзор, введение в тему, обозначение места данной работы в
% мировых исследованиях и~т.\:п., можно использовать ссылки на~другие
% работы\ifnumequal{\value{bibliosel}}{1}{~\autocite{Gosele1999161}}{}
% (если их~нет, то~в~автореферате
% автоматически пропадёт раздел <<Список литературы>>). Внимание! Ссылки
% на~другие работы в разделе общей характеристики работы можно
% использовать только при использовании \verb!biblatex! (из-за технических
% ограничений \verb!bibtex8!. Это связано с тем, что одна
% и~та~же~характеристика используются и~в~тексте диссертации, и в
% автореферате. В~последнем, согласно ГОСТ, должен присутствовать список
% работ автора по~теме диссертации, а~\verb!bibtex8! не~умеет выводить в одном
% файле два списка литературы).
% При использовании \verb!biblatex! возможно использование исключительно
% в~автореферате подстрочных ссылок
% для других работ командой \verb!\autocite!, а~также цитирование
% собственных работ командой \verb!\cite!. Для этого в~файле
% \verb!Synopsis/setup.tex! необходимо присвоить положительное значение
% счётчику \verb!\setcounter{usefootcite}{1}!.

% Для генерации содержимого титульного листа автореферата, диссертации
% и~презентации используются данные из файла \verb!common/data.tex!. Если,
% например, вы меняете название диссертации, то оно автоматически
% появится в~итоговых файлах после очередного запуска \LaTeX. Согласно
% ГОСТ 7.0.11-2011 <<5.1.1 Титульный лист является первой страницей
% диссертации, служит источником информации, необходимой для обработки и
% поиска документа>>. Наличие логотипа организации на титульном листе
% упрощает обработку и поиск, для этого разметите логотип вашей
% организации в папке images в формате PDF (лучше найти его в векторном
% варианте, чтобы он хорошо смотрелся при печати) под именем
% \verb!logo.pdf!. Настроить размер изображения с логотипом можно
% в~соответствующих местах файлов \verb!title.tex!  отдельно для
% диссертации и автореферата. Если вам логотип не~нужен, то просто
% удалите файл с логотипом.

{\actuality} 
Начиная с зарождения первых цивилизаций и до сегодняшнего момента вопросы
о том, как устроены сознание и разум в человеке и других живых существах, продолжают быть ключевыми для человечества.
Так сложилось, что их осмысление происходило сперва в рамках религиозной,
а затем и философской парадигмы, которые в западноевропейской традиции значительно перекликались и дополняли друг друга.
От платоновского мира идей и парменидовского разграничения истины и мнения о ней через
средневековое понятие об истине как о благодати, исходящей от Бога к tabula rasa эмпириков
и критике чистого разума Канта и далее вплоть до языковых игр Витгенштейна эволюционировало
представление о том, как устроено познание человеком окружающей действительности.

Научный подход к вопросам познания сформировался лишь относительно недавно в рамках группы
дисциплин, включающей в себя нейробиологию, нейрофизиологию, когнитивную психологию и 
электрофизиологию. В рамках этих дисциплин ключевым для понимания когнитивной функции человека
и животных в самом широком смысле становится устройство центральной нервной системы и, в частности,
головного мозга.

Интересно, что механизмы познания связывались с функцией головного мозга не всегда даже
в рамках материалистического описания. Так например, Аристотель считал источником мысли сердце,
а мозгу отводил лишь роль радиатора, охлаждающего кровь. Тем не менее, уже в эпоху классической
античности Гален сформировал идею о том, что именно мозг является источником мысли,
а значит и тем инструментом, с помощью которого реализуется познание.
На укоренение идеи о том, что изучение работы центральной нервной системы и её высшего отдела
--- коры больших полушарий --- способно предложить ответ на фундаментальный вопрос
“что представляет собой человеческий интеллект”, ушло еще более полутора тысяч лет. 
На сегодняшний день удовлетворительного ответа на этот вопрос по-прежнему нет и появится он,
вероятно, не скоро. Однако в ходе  долгого и непростого движения к этой Ultima Thule
наше понимание более прикладных вещей, вращающихся около нейрофизиологии, несравненно обогатилось.
С практической точки зрения трудно переоценить значение понимания работы ЦНС для медицины, не ограничиваясь, однако, лишь ею. 

Сегодня, вместе со всеобщим размыванием междисциплинарных границ, нейронауки все больше оказываются
связанными с более техническими и инженерно-прикладными дисциплинами.
Так, в 1943 году вдохновленные архитектурой нейронных ансамблей живого мозга Маккаллок и Питтс
создают первую вычислительную модель нейронной сети, породив тем самым столь популярный сегодня
класс алгоритмов машинного обучения. Все большую популярность приобретают сегодня мозг-компьютерные интерфейсы,
позволяющие формировать управляющую команду на основе электромагнитной активности мозга напрямую,
что открывает совершенно новые перспективы для интеграции человека с машиной.

В этой связи развитие методов, связанных с изучением строения и работы мозга а также декодирование
порождаемых им сигналов представляет сегодня чрезвычайный интерес.
Вместе с тем, за последние сто лет благодаря резкому скачку в развитии электроники,
физики и компьютерных наук набор инструментов в руках ученого-нейрофизиолога существенно обогатился.
На сегодняшний день существующие методы с точки зрения необходимости хирургического вмешательства
для проведения измерений можно разделить на инвазивные и неинвазивные.

К первой группе относится интракраниальная энцефалография --- метод,
в котором электрические потенциалы записываются напрямую с коры больших полушарий.
Недостатки и преимущества такого подхода очевидны. К первым прежде всего относится необходимость
хирургического вмешательства для проведения измерений, что существенно ограничивает возможности
исследователя-нейрофизиолога в получении данных для исследования.
На практике осуществление таких измерений на человеке возможно лишь для пациентов,
прошедших операцию на мозге в связи с каким-либо неврологическим заболеванием, как правило эпилепсией.
В ходе операции для мониторинга активности мозга после хирургического вмешательства на кору головного
мозга устанавливаются электроды, регистрирующие электрическую активность.
Ясно, что количество таких данных, как и  возможность проведения каких-либо сложных
когнитивных экспериментов на пациентах, прошедших операцию на мозге, весьма ограничены.
При этом качество электрического сигнала, записанного в непосредственной близости от его источника,
несравненно выше того, что можно получить, записывая электроэнцефалограмму с поверхности кожи головы.

Неинвазивные методы, с другой стороны, представляют собой намного более гибкий инструмент
для исследований головного мозга человека в силу отсутствия необходимости проведения операции.
Для изучения анатомической организации мозга а также в качестве вспомогательного инструмента
при анализе активности нейронных популяций коры используются методы структурной нейровизуализации,
такие как магнитно-резонансная томография (МРТ),
компьютерная томография (КТ) и диффузионная тензорная визуализация (ДТВ).
Они позволяют неинвазивно получать статические трехмерные изображения тканей головного мозга.
Для изучения динамической активности нейронов используются функциональные методы нейровизуализации,
а именно -  функциональная магнитно-резонансная томография (фМРТ),
позитронно-эмиссионная томография (ПЭТ), электроэнцефалогафия (ЭЭГ), а также магнитная энцефалография (МЭГ).

При этом лишь последние два метода измеряют электрическую активность мозга непосредственно,
тогда как фМРТ и ПЭТ меряют локальный кровоток, что определяет временное разрешение метода. 
Для ЭЭГ и МЭГ временное разрешение оказывается равным $\approx 1 мс$, тогда как методы,
измеряющие локальный кровоток, позволяют разрешить лишь процессы с характерными временами 
порядка одной секунды и медленнее. 

% \ifsynopsis
% Этот абзац появляется только в~автореферате.
% Для формирования блоков, которые будут обрабатываться только в~автореферате,
% заведена проверка условия \verb!\!\verb!ifsynopsis!.
% Значение условия задаётся в~основном файле документа (\verb!synopsis.tex! для
% автореферата).
% \else
% Этот абзац появляется только в~диссертации.
% Через проверку условия \verb!\!\verb!ifsynopsis!, задаваемого в~основном файле
% документа (\verb!dissertation.tex! для диссертации), можно сделать новую
% команду, обеспечивающую появление цитаты в~диссертации, но~не~в~автореферате.
% \fi

% {\progress} 
% Этот раздел должен быть отдельным структурным элементом по
% ГОСТ, но он, как правило, включается в описание актуальности
% темы. Нужен он отдельным структурным элементом или нет ---
% смотрите другие диссертации вашего совета, скорее всего не нужен.

{\aim} данной работы является \ldots

Для~достижения поставленной цели необходимо было решить следующие {\tasks}:
\begin{enumerate}
  \item Исследовать, разработать, вычислить и~т.\:д. и~т.\:п.
  \item Исследовать, разработать, вычислить и~т.\:д. и~т.\:п.
  \item Исследовать, разработать, вычислить и~т.\:д. и~т.\:п.
  \item Исследовать, разработать, вычислить и~т.\:д. и~т.\:п.
\end{enumerate}


{\novelty}
\begin{enumerate}
  \item Впервые \ldots
  \item Впервые \ldots
  \item Было выполнено оригинальное исследование \ldots
\end{enumerate}

{\influence} \ldots

{\methods} \ldots

{\defpositions}
\begin{enumerate}
  \item Первое положение
  \item Второе положение
  \item Третье положение
  \item Четвертое положение
\end{enumerate}
В папке Documents можно ознакомиться в решением совета из Томского ГУ
в~файле \verb+Def_positions.pdf+, где обоснованно даются рекомендации
по~формулировкам защищаемых положений. 

{\reliability} полученных результатов обеспечивается \ldots \ Результаты находятся в соответствии с результатами, полученными другими авторами.


{\probation}
Основные результаты работы докладывались~на:
перечисление основных конференций, симпозиумов и~т.\:п.

{\contribution} Автор принимал активное участие \ldots

%\publications\ Основные результаты по теме диссертации изложены в ХХ печатных изданиях~\cite{Sokolov,Gaidaenko,Lermontov,Management},
%Х из которых изданы в журналах, рекомендованных ВАК~\cite{Sokolov,Gaidaenko}, 
%ХХ --- в тезисах докладов~\cite{Lermontov,Management}.

\ifnumequal{\value{bibliosel}}{0}{% Встроенная реализация с загрузкой файла через движок bibtex8
    \publications\ Основные результаты по теме диссертации изложены в XX печатных изданиях, 
    X из которых изданы в журналах, рекомендованных ВАК, 
    X "--- в тезисах докладов.%
}{% Реализация пакетом biblatex через движок biber
%Сделана отдельная секция, чтобы не отображались в списке цитированных материалов
    \begin{refsection}[vak,papers,conf]% Подсчет и нумерация авторских работ. Засчитываются только те, которые были прописаны внутри \nocite{}.
        %Чтобы сменить порядок разделов в сгрупированном списке литературы необходимо перетасовать следующие три строчки, а также команды в разделе \newcommand*{\insertbiblioauthorgrouped} в файле biblio/biblatex.tex
        \printbibliography[heading=countauthorvak, env=countauthorvak, keyword=biblioauthorvak, section=1]%
        \printbibliography[heading=countauthorconf, env=countauthorconf, keyword=biblioauthorconf, section=1]%
        \printbibliography[heading=countauthornotvak, env=countauthornotvak, keyword=biblioauthornotvak, section=1]%
        \printbibliography[heading=countauthor, env=countauthor, keyword=biblioauthor, section=1]%
        \nocite{%Порядок перечисления в этом блоке определяет порядок вывода в списке публикаций автора
                vakbib1,vakbib2,%
                confbib1,confbib2,%
                bib1,bib2,%
        }%
        \publications\ Основные результаты по теме диссертации изложены в~\arabic{citeauthor}~печатных изданиях, 
        \arabic{citeauthorvak} из которых изданы в журналах, рекомендованных ВАК, 
        \arabic{citeauthorconf} "--- в~тезисах докладов.
    \end{refsection}
    \begin{refsection}[vak,papers,conf]%Блок, позволяющий отобрать из всех работ автора наиболее значимые, и только их вывести в автореферате, но считать в блоке выше общее число работ
        \printbibliography[heading=countauthorvak, env=countauthorvak, keyword=biblioauthorvak, section=2]%
        \printbibliography[heading=countauthornotvak, env=countauthornotvak, keyword=biblioauthornotvak, section=2]%
        \printbibliography[heading=countauthorconf, env=countauthorconf, keyword=biblioauthorconf, section=2]%
        \printbibliography[heading=countauthor, env=countauthor, keyword=biblioauthor, section=2]%
        \nocite{vakbib2}%vak
        \nocite{bib1}%notvak
        \nocite{confbib1}%conf
    \end{refsection}
}
При использовании пакета \verb!biblatex! для автоматического подсчёта
количества публикаций автора по теме диссертации, необходимо
их~здесь перечислить с использованием команды \verb!\nocite!.
