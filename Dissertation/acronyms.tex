\chapter*{Список сокращений и условных обозначений}             % Заголовок
\addcontentsline{toc}{chapter}{Список сокращений и условных обозначений}  % Добавляем его в оглавление
\noindent
%\begin{longtabu} to \dimexpr \textwidth-5\tabcolsep {r X}
\begin{longtabu} to \textwidth {r X}
% Жирное начертание для математических символов может иметь
% дополнительный смысл, поэтому они приводятся как в тексте
% диссертации
    $\mu$  & магнитная проницаемость в вакууме\\
    $\matr{G}$ & матрица прямой модели\\

    \textbf{BEM} & boundary element method, метод граничных элементов\\
    \textbf{ЦНС} & центральная нервная система\\
    \textbf{ЭЭГ} & электроэнцефалография\\
    \textbf{МЭГ} & магнитная электроэнцефалография\\
    \textbf{МРТ} & магнитнo-резонансная томография\\
    \textbf{фМРТ} & функциональная магнитнo-резонансная томография\\
    \textbf{КТ} & компьютерная томография\\
    \textbf{ДТВ} & диффузионно-тензорная визуализация\\
    \textbf{ПЭТ} & позитронно-эмиссионная томография\\
    \textbf{CTC} & communication through coherence; взаимодействие через когерентность\\
    \textbf{PLI} & phase lag index; индекс фазовой задержки\\
    \textbf{wPLI} & weighted phase lag index; взвешенный индекс фазовой задержки\\
    \textbf{ОСШ} & отношение сигнал / шум\\
    \textbf{GCS} & geometric correction scheme\\
    \textbf{DICS} & dynamic imaging of coherent sources\\
    \textbf{MNE} & minimum norm estimate\\
    \textbf{MCE} & minimum current estimate\\
    \textbf{ISTA} & iterative shrinkage thresholding algorithm\\
    \textbf{MxNE} & mixed-norm estimate\\
    \textbf{irMxNE} & iteratve reweighted mixed-norm estimate\\
    \textbf{PSF} & point spread function\\
    \textbf{RK} & resolution kernel\\
    \textbf{BR} & beam response\\
    \textbf{ROC} & Receiver Operating Characteristics\\
    \textbf{PR} & Precision-Recall\\
    \textbf{БИХ-фильтр} & фильтр с бесконечной импульсной характеристикой\\
    \textbf{КИХ} & фильтр с конечной импульсной характеристикой
\end{longtabu}
\addtocounter{table}{-1}% Нужно откатить на единицу счетчик номеров таблиц, так как предыдующая таблица сделана для удобства представления информации по ГОСТ
