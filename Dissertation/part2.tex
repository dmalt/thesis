
% \section{Введение}

\section{Использование методов решения обратной задачи для оценки фазовой синхронности.}
Как мы уже отмечали в \ref{TODO}, оценка коннективности в пространстве сенсоров
имеет два недостатка:
\begin{itemize}
    \item перемешивание сигнала на сенсорах искуственно увеличивает коннективность
        а также ведёт к неправильной оценке мощностей.
    \item анализ сенсоров даёт менее пространственно специфичную картину активности.
\end{itemize}

Для оценки коннективности на уровне сенсоров методы, рассмотренные в предыдущем разделе, могут быть использованы
напрямую, как часть двухэтапной процедуры. Сначала при помощи одного из методов решения обратной задачи
оцениваются временные ряды источников. Затем для источников применяется одна из мер коннективности.

Такая двухшаговая процедура является распространенным способом
оценки коннективности на источниках \cite{TODO:add some refs}. Тем не менее,
она имеет один важный недостаток: так как метод решения ОЗ % TODO


\subsection{DICS}
\label{DICS_subsection}
Рассмотрим теперь одну важную модификацию метода векторных бимформеров,
разработанную для анализа коннективностей~\cite{DICS}.

Суть метода состоит в построении адаптивных фильтров аналогично тому, как это делалось 
в разделе~\ref{sec:min_interference_filter}, но на этот раз в частотной области.

Рассмотрим подробно этапы построения DICS-фильтров.
Начнем вновь с рассмотрения порождающей модели сигнала
\begin{equation}
    \mathbf{X} = \mathbf{G} \mathbf{S} + \mathbf{\Omega},
    \label{gm_dics}
\end{equation}


Применим к данным частотно-временное преобразование и посчитаем кросс-спектр по аналогии
с~\ref{gm_timefreq} и~\ref{gm_cp_matr}:

\begin{gather}
    \Cp{x} = \mathbf{G} \Cp{s} \mathbf{G}^T + \Cp{\omega}
    \label{gm_cp_matr}
\end{gather}

Оптимизационная задача метода DICS формализуется следующим образом:

\begin{equation}
    \tr\left\{\mathbf{A}_k \Cp{X} \mathbf{A}_k^T\right\} +
     \alpha \tr\left\{\mathbf{A}_k \mathbf{A}_k^T\right\}
    \rightarrow \underset{\mathbf{A}_k}{\min}
    , s.t.: \mathbf{A}_k \mathbf{G}_k = \mathbf{I}
\end{equation}

Оптимизационный критерий аналогичен критерию из раздела \ref{sec:min_interference_filter}
для векторной постановки задачи адаптивной фильтрации,
минимизирующей вклад от третих источников, c добавлением регуляризации матрицы кросс-спектра.

Проводя покомпонентный вывод методом множителей Лагранжа по аналогии с
выводом формул~\ref{lcmv_filters_vec_comp},~\ref{lcmv_filters_vec}, получим
выражения для фильтров в следующем виде:

\begin{equation}
    \mathbf{A}_k =
    {\left(\mathbf{G}_k^T {(\Cp{X} + \alpha \mathbf{I})}^{-1} \mathbf{G}_k\right)}^{-1}
    \mathbf{G}_k^T {\left(\Cp{X} + \alpha \mathbf{I}\right)}^{-1}
\end{equation}

Для получения оценок на элементы матрицы кросс-спектральной плотности для источников
умножим кросс-спетр измерений на полученные матрицы фильтров:

\begin{equation}
    \mathbf{C}^{loose}_{kl} = \mathbf{A}_k \Cp{X} \mathbf{A}_l^H
    \label{eq:dics_c_loose}
\end{equation}

Полученная матрица $3\times 3$, $\mathbf{C}^{loose}_{kl}$
содержит оценки элементов кросс-спектра источников по трем ортогональным направлениям.
Оценочное значения элемента кросс-спектра для истинной ориентации диполя
выбирается как след матрицы $\mathbf{C}^{loose}_{kl}$:

\begin{equation}
    \Cp{S}_{kl} = \tr\left\{\mathbf{A}_k \Cp{X} \mathbf{A}_l^H\right\}
\end{equation}
В случае, если максимальное собственное значение матрицы $\mathbf{C}^{loose}_{kl}$
сильно превосходит два других, в качестве оценки на $\Cp{S}_{kl}$ можно брать
максимальное собственное число матрицы $\mathbf{C}^{loose}_{kl}$, что соответствует
случаю, когда направления диполей в точках $k$ и $l$ приближенно оставались неизменными
за время оценки матрицы $\Cp{X}$.

Окончательно, для двух источников $k$, $l$ будем иметь оценку квадрата когерентности вида

\begin{equation}
    \coh{(k, l)}^2 = \frac{\Cp{S}_{kl}}{\sqrt{\Cp{S}_{kk} \Cp{S}_{ll}}}
\end{equation}

Отметим, что такой метод оценки функциональной коннективности, хотя и является на сегодняшний
день одним из самых популярных в нейронаучном сообществе, никак не борется с проблемой
протечки сигнала, а следовательно подвержен ложноположительным срабатываниям и сложностью
в аккуратной интерпретации полученного распределения коннективнотей.

\subsection{iDICS}
Чтобы сделать метод DICS устойчивым к протечке сигнала, мы можем воспользоваться
идеей метода мнимой когерентности~\ref{sec:imcoh}. Для этого при оценке внедиагональных
элементов матрицы кросс-спектра в пространстве источников будем умножать DICS фильтры
не на весь кросс-спектр на сенсорах, а только на его мнимую часть.

Такая модификация метода DICS предоставляет сравнительно простой способ
оценки мнимой части матрицы кросс-спектральной плотности в пространстве
источников. Использование мнимой части, как и для метода imcoh, позволяет
устранить эффект протечки сигнала при оценке коннективности.

Формально метод iDICS получается модификацией формулы~\ref{eq:dics_c_loose}:
вместо полного кросс-спектра на сенсорах необходимо взять его мнимую часть:

\begin{equation}
    \mathbf{C}^{loose}_{kl} = \mathbf{A}_k \Im(\Cp{X}) \mathbf{A}_l^H
    \label{eq:idics_c_loose}
\end{equation}

Чтобы фильтры были действительнозначными, при их построении
можно пользоваться только действительной частью кросс-спектра на источниках:
\begin{equation}
    \mathbf{A}_k =
    {\left(\mathbf{G}_k^T {(\Re(\Cp{X}) + \alpha \mathbf{I})}^{-1} \mathbf{G}_k\right)}^{-1}
    \mathbf{G}_k^T {\left(\Re(\Cp{X}) + \alpha \mathbf{I}\right)}^{-1}
    \label{eq:idics_filters}
\end{equation}

Вычисление когерентности на источниках требует нормализации оцененного
внедиагонального элемента кросс-спектра на мощности взаимодействующих
источников. Использование мнимой части кросс-спектра не позволяет
оценить мощности источников, поэтому для оценки мощностей необходимо либо
пользоваться действительной частью кросс-спектра с теми же действительнозначными
фильтрами.


\subsection{GCS}
Оценка методом мнимой когерентности, а также методы, основанные на 
той же идее, имеют один существенный недостаток: они не позволяет
детектировать сети с близкой к нулю фазовой задержкой. Это наблюдение
заставляет искать иные способы борьбы с методом протечки сигнала.

Одним из таких методов является метод геометрической коррекции
(GCS),~\cite{GCS}.  Идея метода основана на том наблюдении, что протечка
сигнала из точки Б коры в точку А пропорциональна скалярному произведению
топографии точки Б на фильтр, восстанавливающий сигнал в точке А. Метод
геометрической коррекции предлагает для оценки когерентности между точками А и
Б удалять из сигнала источник с топографией как у точки Б.

Оценка методом GCS включает четыре шага:

\begin{enumerate}
    \item спроецировать вектор сигнала на сенсорах ортогонально
        топографии точки Б.
    \item на основе спроецированных данных одним из методов решения
        обратных задач оценить активность в точке А.
    \item использовать исходный, неспроецированный вектор сигнала на на сенсорах
        для оценки активности в точке Б.
    \item посчитать какую-либо меру коннективности между активностями
        оцененными в точках А и Б (coh, plv и т.д.)
\end{enumerate}

Фактически метод геометрической коррекции предлагает использовать вместо идеи
imcoh пространственную фильтрацию, которая устраняет протечку сигнала между
точками, в которых мы хотим померить коннективность.

К достоинствам этого метода относится прежде всего сохранение в сигнале
информации о коннективности с нулевой фазовой задержкой. Кроме того, метод
является сравнительно простым в вычислительном плане.

К недостаткам метода относится в первую очередь тот факт, что метод
геометрической коррекции при проекции игнорирует протечку сигнала от всех
остальных источников, кроме пары, для которой считается коннективность. Это
приводит к тому, что протечка сигнала от третих источников в точки А и Б все
равно вызывает ложную коннективноть.  Иными словами, метод GCS полностью
избавляется от ложной коннективности, вызванной протечкой сигнала только в том
случае, если активно всего два источника, а шум мозга отсутствует.

\subsection{PSIICOS}
Идея использования пространственной фильтрации в качестве метода борьбы с
протечкой сигнала от совокупности всех активных источников была рассмотрена
нами в разделе~\ref{sec:TODO}.

Также в разделе~\ref{sec:TODO} мы рассмотрели существующие методы решения
обратной задачи для МЭЭГ

Теперь мы можем использовать PSIICOS-проекцию для оценки
коннективности на уровне источников.

Процедура поиска источников состоит в следующем:

\begin{enumerate}
    \item Находим кросс-спектр на уровне сенсоров
    \item По посчитанной матрице прямой модели строим оператор проекции от протечки сигнала (PSIICOS-проекция)
    \item Векторизуем кросс-спектр на сенсорах и применяем к результату PSIICOS-проекцию
    \item По векторизованному кросс-спектру на сенсорах ищем элементы кросс-спектра на
        источниках, используя фильтр, максимизирующий SNR в данной точке~\ref{sec:max_snr_filter}.
\end{enumerate}

Для построения фильтра, дающего оценку кросс-спектрального коэффициента в точке
$i, j$, мы берем топографии соответствующих точек $\vec{g}_i$, $\vec{g}_j$ и
строим по ним 2-топографию $\vec{g}_j \otimes \vec{g}_i$. Затем мы применям
к этой топографии PSIICOS-проекцию и нормируем к единице полученный вектор.
Результат и будет искомым фильтром. Применение этого фильтра к векторизованному кросс-спектру
дает оценку кросс-спектрального коэффициента $i, j$ в пространстве источников.

Такая процедура предоставляет работающую на практике эвристику без строгого обоснования.
Вместе с тем, существует более общий взгляд на операцию проекции от протечки сигнала,
позволяющий получить описанный выше алгоритм более формально, а также дающий возможность
сформулировать обобщения алгоритма PSIICOS.

\subsubsection{PSIICOS как метод оптимальной фильтрации.}

Протечка сигнала является главной проблемой при оценке коннективности при
условии, что мы не хотим пренебрегать частью сигнала, содержащей информацию
о коннективности с малыми задержками по фазе.

Важно понимать, что эта протечка не ограничивается проникновением друг в друга
сигналов только между двумя источниками, для которых мы измеряем
коннективность. Любой источник, протекающий \emph{одновременно} в эти два приведет
к появлению ложной коннективности.

Чтобы правильно оценить коннективность между парой источников,
необходимо построить такой набор фильтров, который максимально
подавлял бы источники, дающие большую протечку одновременно в два
целевых источнка. Построим такой фильтр.

В теории обратных задач вводят понятие ядра разрешения $R(\vec{r},
\vec{r}^{\prime})$ (resolution kernel, RK~\cite{sekikhara_nagarajan}): возьмем
обратный оператор, восстанавливающий по измерениям активность в точке коры
$\vec{r}$, и зафиксируем другую точку на коре, $\vec{r}^{\prime}$.  Функция
$R(\vec{r}, \vec{r}^{\prime})$ показывает, сколько сигнала протекает из точки
$\vec{r}^{\prime}$ в точку $\vec{r}$ при оценке активности в точке $\vec{r}$.
Для линейных методов решения обратной задачи функция $R$ может быть записана как
скалярное произведение:
\begin{equation}
    R(i, k) = \vec{l}_i^T \vec{g}_k
    \label{eq:resolution_kernel}
\end{equation}
Здесь $\vec{l}_i$~--- фильтр в направлении $i$-ой точки коры, а
$\vec{g}_j$~--- топография точки коры с индексом $k$.

Для совокупности всех $N$ точек коры можно определить векторнозначную функцию
$\vec{B}(i)$, которая имеет смысл протечки сигнала от каждой точки коры в
фиксированную точку $i$. 
\begin{equation}
    \vec{B}{(i)}^T = \left[R{(i, 1)}, R(i, 2), \ldots, R(i, N)\right]
\end{equation}
Эта функция получила в литературе название Beam Response (BR)~\cite{sekikhara_nagarajan}.


Зафиксируем теперь две точки коры $i$, $j$. Для этих точек степень перекрытия
векторов $\vec{B}(i)$ и $\vec{B}(j)$ определяет, насколько мощной будет общая
для двух точек компонента нежелательного сигнала. Так, если $\vec{B}(i)$ и
$\vec{B}(j)$ имеют только неотрицательные компоненты и при этом ортогональны
друг другу, общий сигнал, протекающий одновременно и в $i$ и в $j$ будет равен
нулю. В этом случае оценка коннективности будет полностью свободна от
негативного эффекта протечки сигнала. Отсюда возникает желание построить такой
набор фильтров, который минимизирует перекрытие $\vec{B}(i)$ и $\vec{B}(j)$ для
каждой пары точек.

Для построения таких фильтров необходимо определить, как измерять перекрытие
для векторов BR. В качестве меры перекрытия напрашивается скалярное
произведение векторов $\vec{B}(i)$, $\vec{B}(j)$: если оно равно нулю, т.е.
вектора ортогональны, общая протечка отсутствует. Такая мера перекрытия,
однако, является ошибочной в силу возможной знакопеременности компонент
векторов $\vec{B}(i)$, $\vec{B}(j)$. Например, в случае, когда есть всего два
источника, а функции $\vec{B}$ для них равны $(1, -1)$ и $(1, 1)$, формально BR
будут ортогональны, однако протечка сигнала между соответствующими источниками
будет весьма значительной.

Чтобы преодолеть проблему знакопеременных компонент, будем измерять степень
перекрытия как скалярное произведение векторов $\vec{B}(i)$, $\vec{B}(j)$,
покомпонентно возведённых в квадрат. Результат
этого скалярного произведения, функцию $\mu(i, j)$, назовём \emph{взаимной
протечкой сигнала} для точек $i$, $j$:
\begin{equation}
    \mu(i, j) \defeq {(\vec{B}{(i)}^{\odot 2})}^T \vec{B}{(j)}^{\odot 2}
    \label{eq:def_mutual_leakage}
\end{equation}
Здесь операция ${(\cdot)}^{\odot 2}$ означает поэлементное возведение в квадрат.

С учетом соотношений~\ref{eq:resolution_kernel},~\ref{eq:kron_vec} взаимную протечку сигнала можно переписать как
\begin{multline}
    \mu(i, j)= \sum_k {\left(\vec{l}_i^T \vec{g}_k\right)}^2{\left(\vec{l}_j^T \vec{g}_k\right)}^2=
    \sum_k\left(\vec{l}_i^T \vec{g}_k\vec{g}_k^T \vec{l}_j\right)^2=(\vec{l}_j \otimes \vec{l}_i)^T \matr{\Gamma} \matr{\Gamma}^T(\vec{l}_j \otimes \vec{l}_i),\\
    \matr{\Gamma} = (\vec{g}_1 \otimes \vec{g}_1, \ldots, \vec{g}_N \otimes \vec{g}_N)
\end{multline}

Мы хотим найти такую пару фильтров $\vec{l}_i, \vec{l}_j$, которая
минимизировала бы взаимную протечку сигнала для точек $i, j$, сохраняя при этом
полезный сигнал. Для этого необходимо дополнительно наложить ограничение на
длину фильтров. Одним из возможных ограничений является требование, чтобы
коэффициент усиления сигнала, приходящего из точек с топографиями $\vec{g}_i,
\vec{g}_j$ равнялся единице. С учетом этого ограничения оптимизационная задача
на поиск пары фильтров $\vec{l}_i, \vec{l}_j$ запишется как
\begin{gather}
    \frac{1}{2} (\vec{l}_j \otimes \vec{l}_i)^T \matr{\Gamma} \matr{\Gamma}^T(\vec{l}_j \otimes \vec{l}_i) \rightarrow \min\\
    s.t.: \vec{l}_i^T \vec{g}_i = 1, \, \vec{l}_j^T \vec{g}_j = 1
    \label{eq:min_leakage_objective}
\end{gather}

Такая задача оптимизации не разрешается в явном виде методом множителей
Лагранжа из-за произведения Кронекера в целевом функционале. Ее необходимо
решать численно, например методом градиентного спуска. Тем не менее, эту задачу
можно значительно упростить, обобщив понятие фильтров на пространство размерности $M^2$.

Отметим, что из ограничений задачи \ref{eq:min_leakage_objective} следует, что
\begin{equation}
    (\vec{l}_j \otimes \vec{l}_i)^T (\vec{g}_j \otimes \vec{g}_i) = 1
    \label{eq:min_leakage_restriction_kron}
\end{equation}

Теперь воспользуемся тем, что ограничение, записанное в таком виде, как и
целевая функция в \ref{eq:min_leakage_objective} зависят от фильтров
$\vec{l}_i, \vec{l}_j$ только через их кронекеровское произведение.
Обозначим это произведение как $\vec{v}_{ij}$.
Для простоты также обозначим $\vec{g}_j \otimes \vec{g}_i$ как $\vec{q}_{ij}$.
\begin{gather}
    \frac{1}{2} \vec{v}_{ij}^T \matr{\Gamma} \matr{\Gamma}^T\vec{v}_{ij} \rightarrow \min\\
    s.t.: \vec{v}_{ij}^T \vec{q}_{ij} = 1
    \label{eq:min_leakage_objective_kron}
\end{gather}

Такая задача относительно переменной $\vec{v}_{ij}$ легко решается методом множителей Лагранжа.
Ее лагранжиан запишется как
\begin{equation}
    L(\vec{v}_{ij}, \lambda) = \frac{1}{2} \vec{v}_{ij}^T \matr{\Gamma} \matr{\Gamma}^T\vec{v}_{ij} + \lambda (\vec{v}_{ij}^T \vec{q}_{ij} - 1)
    \label{eq:min_leakage_lagrangian}
\end{equation}

Дифференцируя лагранжиан по $\vec{v}_{ij}$, получим
\begin{equation}
    \frac{\partial L(\vec{v}_{ij}, \lambda)}{\partial \vec{v}_{ij}} = \matr{\Gamma} \matr{\Gamma}^T\vec{v}_{ij} + \lambda \vec{q}_{ij} = 0
    \label{eq:min_leakage_lagrangian_diff_v}
\end{equation}

Если $\lambda \neq 0$, уравнение \ref{eq:min_leakage_lagrangian_diff_v} не имеет решений:
столбцы матрицы $\matr{\Gamma}$ являются векторизациями симметричных матриц, а
значит таковым является и произведение $\matr{\Gamma} \matr{\Gamma}^T\vec{v}_{ij}$; при этом
$\vec{q}_{ij}$ векторизацией симметричной матрицы не является. Это означает, что $\lambda = 0$,
и уравнение на $\vec{v}_{ij}$ выглядит как 
\begin{equation}
    \matr{\Gamma} \matr{\Gamma}^T\vec{v}_{ij} = 0
\end{equation}

Следовательно что вектор $\vec{v}_{ij}$ должен принадлежать ортогональному
дополнению линейной оболочки столбцов матрицы $\matr{\Gamma}$. Это значит, что
вектор $\vec{v}_{ij}$ можно получить как результат применения проекции
$\matr{P} = \matr{I} - \matr{\Gamma} \matr{\Gamma} ^ {\dagger}$
к некоторому вектору $\vec{\tilde{v}}_{ij}$ из общего линейного пространства размерностью $M^2$.

Продифференцируем теперь лагранжиан \ref{eq:min_leakage_lagrangian} по $\lambda$:
\begin{equation}
    \frac{\partial L(\vec{v}_{ij}, \lambda)}{\partial \lambda} = \vec{v}_{ij}^T \vec{q}_{ij} - 1 = 0
    \label{eq:min_leakage_lagrangian_diff_lambda}
\end{equation}
С учетом общего вида вектора $\vec{v}_{ij}$, а также идемпотентности и
симметричности оператора проекции получим
\begin{equation}
    \vec{\tilde{v}}_{ij}^T \matr{P}^T \vec{q}_{ij} = 
    \vec{\tilde{v}}_{ij}^T \matr{P} \vec{q}_{ij} =
    \vec{\tilde{v}}_{ij}^T \matr{P}^2 \vec{q}_{ij} =
    (\matr{P}\vec{\tilde{v}}_{ij})^T \matr{P} \vec{q}_{ij} =
    \vec{v}_{ij}^T \matr{P} \vec{q}_{ij} = 1
\end{equation}

Последнее уравнение задает гиперплоскость в подпространстве векторов, ортогональных
столбцам $\matr{\Gamma}$. Следовательно поставленная задача оптимизации имеет бесконечное
количество решений. Для выбора единственного решения можем как и раньше воспользоваться
критерием минимальности нормы.
Среди векторов этой гиперплоскости минимальным по норме вектором
являестся
\begin{equation}
    \vec{v}_{ij} = \frac{\matr{P} \vec{q}_{ij} } { \norm{\matr{P} \vec{q}_{ij}}^2 }
    \label{eq:min_leakage_solution}
\end{equation}

Таким образом получаем, что фильтр, минимизирующий взаимную протечку сигнала
для точек коры $i, j$ получается как результат проекции, полученной нами в
разделе \ref{sec:psiicos_projection}, примененной к соответсвтующей
2-топографии.

Отметим, что оператор проекции $\matr{P}$ действует как проекция на множество,
состоящее из объединения множества векторизованных антисиметричных матриц и
векторизованных симметричных матриц, не принадлежащих линейной оболочке
2-топографий, соответствующих объемной проводимости. В предельном случае, когда
количество точек на коре превосходит величину $M (M + 1) / 2$, равную общему
количеству симметричных матриц размером $M \times M$, применение оператора
$\matr{P}$ полностью обнуляет векторизованные симметричные матрицы и оставляет
антисимметричные матрицы нетронутыми. Фактически это эквивалентно взятию
мнимой части кросс-спектра, так как топографии мнимой части антисимметричны и
не изменяются проекцией.

Вместе с тем при слишком большом количестве источников численная размерность
линейной оболочки столбцов матрицы $\matr{\Gamma}$ будет существенно меньше
взятого количества источников, а наличие шумовых собственных чисел в матрице
$\matr{\Gamma} \matr{\Gamma}^T$ приведет к чрезмерному удалению полезного
сигнала из действительной части кросс-спектра. Поэтому для адекватного удаления
взаимной протечки сигнала лучше искуственно ограничивать ранг проекции,
например фиксируя шумовой уровень для собственных чисел матрицы $\matr{\Gamma}
\matr{\Gamma}^T$. Ограничение ранга проекции фактически эквивалентно
регуляризации функционала, которую мы рассмотрим далее.

Отметим также, что полученный фильтр~\ref{eq:min_leakage_solution} оказывается
невозможно разложить на кронекеровское произведение двух фильтров, действующих
в пространстве размерности $M$. Тем не менее, его применение к матрице
кросс-спектральной протности мощности на сенсорах дает оценку элемента кросс-спектра на
источниках с желаемыми свойствами.

\subsubsection{Пространственная смещенность оценки и нормализация весов.}
\label{sec:psiicos_normalization_and_spatial_bias}

Проанализируем полученные выражения для фильтров~\ref{eq:min_leakage_solution}
с точки зрения пространственной смещенности оценок. Будем считать, что оценка
является несмещенной, если фильтр $\vec{v}_{ij}$ среди всех возможных фильтров
имеет максимум на 2-топографии $\vec{q}_{ij}$. 

Применим произвольный фильтр $\vec{v}_{rs} = \matr{P} \vec{q}_{rs} / \norm{\matr{P} \vec{q}_{rs}}^2$ к
2-топографии $\vec{q}_{ij}$:
\begin{equation}
    \frac{\vec{q}_{rs}^T\matr{P} \vec{q}_{ij}}{\norm{\matr{P}\vec{q}_{rs}}^2} =
    \frac{
        \norm{\matr{P} \vec{q}_{ij}}
    }{
        \norm{\matr{P}\vec{q}_{rs}}
    }
    \cos(\matr{P}\vec{q}_{rs}, \matr{P} \vec{q}_{ij})
\end{equation}

Как видим, применение произвольного фильтра к нашей 2-топографии может давать
величину как меньше 1, так и больше.  При этом целевой фильтр $\vec{v}_{ij}$ на
этой топографии всегда даёт усиление, равное 1, а значит оценка, получаемая при
помощи фильтра~\ref{eq:min_leakage_solution}, является пространственно
смещенной.

Для построения несмещенной оценки можно воспользоваться нормировкой фильтров аналогично той, которая
используется в методе sLORETA (\cite{sLORETA}). Нормированный фильтр будет выглядеть следующим образом:
\begin{equation}
    \vec{v}_{ij} = \frac{\matr{P} \vec{q}_{ij}}{\norm{\matr{P} \vec{q}_{ij}}}
    \label{eq:min_leakage_solution_sloreta}
\end{equation}

При такой нормировке применение фильтра $\vec{v}_{rs} = \matr{P} \vec{q}_{rs}$
к топографии $\vec{q}_{ij}$ по сравнению с фильтром $\vec{v}_{ij}$ будет выглядеть как
\begin{equation}
    \frac{\vec{q}_{rs}^T\matr{P} \vec{q}_{ij}}{\norm{\matr{P}\vec{q}_{rs}}} =
    \norm{\matr{P} \vec{q}_{ij}} \cos(\matr{P}\vec{q}_{rs}, \matr{P} \vec{q}_{ij}) \leq
    \frac{\vec{q}_{ij}^T\matr{P} \vec{q}_{ij}}{\norm{\matr{P}\vec{q}_{rs}}} =
    \norm{\matr{P} \vec{q}_{ij}}
\end{equation}

Таким образом, фильтры вида~\ref{eq:min_leakage_solution_sloreta} дают
пространственно несмещенную оценку, если следовать определению, принятому в
теории обратных задач.

\subsubsection{Действие фильтра на мнимую и действительную часть кросс-спектра.}

Если в случае с решением обратных задач для восстановления активности в
фиксированной точке коры мы имели дело лишь с одной топографией, то теперь, при
восстановлении элемента матрицы кросс-спектральной плотности в пространстве
источников нам необходимо отдельно рассматривать действительную и мнимую часть
кросс-спектра.

При выводе порождающей модели кросс-спектра в разделе~\ref{sec:psiicos_projection}
мы получили, что 2-топографии действительной и
мнимой части кросс-спектра имеют разную структуру: действительная часть состоит
из векторизации симметричных матриц, тогда как мнимая --- из векторизаций
антисимметричных.

Полученный нами фильтр~\ref{eq:min_leakage_solution_sloreta} обладает симметричной
структурой, а значит никак не влияет на мнимую часть кросс-спектра. При этом
когда мы анализировали пространственную смещенность оценки, мы никак не учитывали, что
вместе с 2-топографией $\vec{q}_{ij}$ в действительную часть спектра всегда входит
источник с сопряженным элементом кросс-спектра на источниках и 2-топографией $\vec{q}_{ji}$.

Проанализируем пространственную смещенность оценки полученным фильтром отдельно
для действительной и мнимой 2-топографий элемента $i, j$.

2-топография действительной части запишется как $\vec{q}_{ij} + \vec{q}_{ji}$.
При этом сам вектор $\vec{q}_{ij}$ может быть расписан как
\begin{equation}
\vec{q}_{ij} = \frac{1}{2} (\vec{q}_{ij} + \vec{q}_{ji}) + \frac{1}{2} (\vec{q}_{ij} - \vec{q}_{ji})
\end{equation}

Тогда действие фильтра $\vec{v}_{rs}$ на 2-топографию действительной части для
точки с индексами $i, j$ с учетом ортогональности симметричных и
антисимметричных слагаемых запишется как

\begin{multline}
    \frac{\left(\frac{1}{2} (\vec{q}_{rs} + \vec{q}_{sr}) + \frac{1}{2} (\vec{q}_{rs} - \vec{q}_{sr})\right)^T\matr{P}\left(\vec{q}_{ij} + \vec{q}_{ji}\right)}
    {\sqrt{\left(\frac{1}{2} (\vec{q}_{rs} + \vec{q}_{sr}) + \frac{1}{2}
    (\vec{q}_{rs} - \vec{q}_{sr})\right)^T\matr{P}\left(\frac{1}{2} (\vec{q}_{rs}
    + \vec{q}_{sr}) + \frac{1}{2} (\vec{q}_{rs} - \vec{q}_{sr})\right)}}  =\\
    =\frac{(\vec{q}_{rs} + \vec{q}_{sr})^T \matr{P} (\vec{q}_{ij} + \vec{q}_{ji})}
    {\sqrt{(\vec{q}_{rs} + \vec{q}_{sr})^T\matr{P}(\vec{q}_{rs} + \vec{q}_{sr}) + (\vec{q}_{rs} - \vec{q}_{sr})^T(\vec{q}_{rs} - \vec{q}_{sr})}} =\\
    = \frac{\cos(\matr{P}(\vec{q}_{rs} + \vec{q}_{sr}), \matr{P}(\vec{q}_{ij} + \vec{q}_{ji})) \norm{\matr{P}(\vec{q}_{ij} + \vec{q}_{ji})}}
    {\sqrt{1 + {\norm{\vec{q}_{rs} - \vec{q}_{sr}}^2}/{\norm{\matr{P}(\vec{q}_{rs} + \vec{q}_{sr})}^2}}}
\end{multline}

Как видим, для действительной части максимальное значение для точки $i, j$ не обязательно достигается
на фильтре $\vec{v}_{ij}$.
Получается, что при рассмотрении только действительной части оценка оказывается смещенной.

Можем получить аналогичную формулу для мнимой части:
\begin{multline}
    \frac{\left(\frac{1}{2} (\vec{q}_{rs} + \vec{q}_{sr}) + \frac{1}{2} (\vec{q}_{rs} - \vec{q}_{sr})\right)^T\matr{P}\left(\vec{q}_{ij} - \vec{q}_{ji}\right)}
    {\sqrt{\left(\frac{1}{2} (\vec{q}_{rs} + \vec{q}_{sr}) + \frac{1}{2}
    (\vec{q}_{rs} - \vec{q}_{sr})\right)^T\matr{P}\left(\frac{1}{2} (\vec{q}_{rs}
    + \vec{q}_{sr}) + \frac{1}{2} (\vec{q}_{rs} - \vec{q}_{sr})\right)}}  =\\
    = \frac{\cos(\vec{q}_{rs} - \vec{q}_{sr}, \vec{q}_{ij} - \vec{q}_{ji}) \norm{\vec{q}_{ij} - \vec{q}_{ji}}}
    {\sqrt{1 + {\norm{\vec{q}_{rs} - \vec{q}_{sr}}^2}/{\norm{\matr{P}(\vec{q}_{rs} + \vec{q}_{sr})}^2}}}
\end{multline}
Здесь оценка также оказывается смещенной.

Чтобы избежать этой проблемы, изменим нормировку нашего фильтра. Менять при этом
будем по-разному для действительной и для мнимой части, получая таким образом два различных фильтра:
\begin{gather}
    \vec{v}_{ij}^{Re} = \frac{2\matr{P}\vec{q}_{ij}}{\norm{\matr{P}(\vec{q}_{ij} + \vec{q}_{ji})}}\\
    \vec{v}_{ij}^{Im} = \frac{2\vec{q}_{ij}}{\norm{\vec{q}_{ij} - \vec{q}_{ji}}}
\end{gather}
Для такой нормировки оценка и по действительной и по мнимой частям отдельно является несмещенной.
Несмещенной также будет оценка по мощности кросс-спектрального коэффициента $i, j$, т.е. по корню
из суммы квадратов мнимой и действительной частей.

\subsubsection{Регуляризация.}
Как уже отмечалось, при достаточно большом количестве точек коры проекция, построенная
без ограничения ранга проекции будет удалять слишком большую долю сигнала из действительной части.

Ограничение ранга проекции позволяет справиться с этой проблемой, однако такой подход
не дает возможности однозначно выбрать решение для фильтров: после проекции мы получали
бесконечное множество решений, лежащих в плоскости, ортогональной спроецированной 2-топографии
для точки $i, j$. Выбор решения осуществлялся нами на основе ad hoc эвристики, что фильтр должен обладать
минимально возможной нормой.

Здесь мы рассмотрим альтернативный способ решения этой проблемы.  Способ основан на тихоновской
регуляризации. Его суть состоит в том, что мы сразу закладываем в целевой функционал слагаемое,
штрафующее норму фильтра. Задача оптимизации в этом случае запишется как
\begin{gather}
    \frac{1}{2} \vec{v}_{ij}^T \matr{\Gamma} \matr{\Gamma}^T\vec{v}_{ij}  + C \frac{1}{2} \norm{\vec{v}_{ij}}^2 \rightarrow \min\\
    s.t.: \vec{v}_{ij}^T \vec{q}_{ij} = 1
    \label{eq:min_leakage_objective_kron_regularized}
\end{gather}

Лагранжиан для этой задачи оптимизации будет выглядеть как
\begin{equation}
    L(\vec{v}_{ij}, \lambda) =
    \frac{1}{2} \vec{v}_{ij}^T \matr{\Gamma} \matr{\Gamma}^T\vec{v}_{ij}  + C \frac{1}{2} \norm{\vec{v}_{ij}}^2 + \lambda(\vec{v}_{ij}^T \vec{q}_{ij} - 1)
    \label{eq:min_leakage_regularized_lagrangian}
\end{equation}

Возьмем его производную по $\vec{v}_{ij}$:

\begin{equation}
    \frac{\partial L(\vec{v}_{ij}, \lambda)}{\partial \vec{v}_{ij}} =
    \matr{\Gamma} \matr{\Gamma}^T\vec{v}_{ij} + C\vec{v}_{ij} + \lambda \vec{q}_{ij} = 0
    \label{eq:min_leakage_reqularized_lagrangian_diff_v}
\end{equation}

На этот раз благодаря регуляризации уравнение \ref{eq:min_leakage_reqularized_lagrangian_diff_v} имеет решение для
ненулевого $\lambda$:
\begin{equation}
    \vec{v}_{ij} = -\lambda (\matr{\Gamma} \matr{\Gamma}^T + C\matr{I})^{-1}\vec{q}_{ij}
\end{equation}

Домножая левую и правую часть на $\vec{q}_{ij}$ и учитывая ограничение
$\vec{v}_{ij}^T\vec{q}_{ij} = 1$, получим выражение для $\lambda$:
\begin{equation}
    \lambda = - \frac{1}{\vec{q}_{ij}^T(\matr{\Gamma}\matr{\Gamma}^T + C\matr{I})^{-1}\vec{q}_{ij}} 
\end{equation}

Для фильтров $\vec{v}_{ij}$ будем иметь
\begin{equation}
    \vec{v}_{ij} = \frac{(\matr{\Gamma}\matr{\Gamma}^T + C\matr{I})^{-1}\vec{q}_{ij}}{\vec{q}_{ij}^T(\matr{\Gamma}\matr{\Gamma}^T + C\matr{I})^{-1}\vec{q}_{ij}} 
\end{equation}

Фильтры, дающие пространственно несмещенную оценку, получим аналогично \ref{sec:psiicos_normalization_and_spatial_bias}:

\begin{gather}
    \vec{v}_{ij}^{Re} = \frac{2(\matr{\Gamma}\matr{\Gamma}^T + C\matr{I})^{-1}\vec{q}_{ij}}
    {\sqrt{(\vec{q}_{ij} + \vec{q}_{ji})^T(\matr{\Gamma}\matr{\Gamma}^T + C\matr{I})^{-1}(\vec{q}_{ij} + \vec{q}_{ji})}}\\
    \vec{v}_{ij}^{Im} = \frac{2(\matr{\Gamma}\matr{\Gamma}^T + C\matr{I})^{-1}\vec{q}_{ij}}
    {\sqrt{(\vec{q}_{ij} - \vec{q}_{ji})^T(\matr{\Gamma}\matr{\Gamma}^T + C\matr{I})^{-1}(\vec{q}_{ij} - \vec{q}_{ji})}}
\end{gather}

Для мнимой части в силу ортогональности симметричных и антисимметричных векторов можем переписать фильтр в виде

\begin{equation}
    \vec{v}_{ij}^{Im} = \frac{2}{\sqrt{C}} \frac{\vec{q}_{ij}}{\norm{(\vec{q}_{ij} - \vec{q}_{ji})}}
\end{equation}

\subsubsection{Нормализация кросс-спектральных коэффициентов в пространстве источников.}

Полученные после фильтрации элементы кросс-спектра в пространстве источников нуждаются
в нормализации, так как их величина зависит не только от постоянства разности фаз между
двумя источниками, но и от мощности взаимодействующих источников. Это приводит к тому, что ненормализованный
кросс-спектральный коэффициент для оценки функциональной коннективности дает смещение
в сторону более мощных источников.

Другим негативным эффектом отсутствия нормализации является тот факт, что
по абсолютной величине кросс-спектрального коэффициента нельзя сделать вывод
о постоянстве разности фаз. С другой стороны нормированная величина позволяет
определить порог, по которому можно судить о наличии или отсутствии фазового взаимодействия.

В чем сложность нормализации при использовании метода PSIICOS?\@
При оценке когерентности используется нормализация на мощность источников. При
использовании PSIICOS-проекции нормализация на мощность оказывается
невозможной, так как мощностная компонента, присутствующая в диагональной части
кросс-спектра, удаляется из сигнала. Если же оценивать мощность по неспроецированным
данным, мощностная компонента, сильно загрязненная протечкой сигнала, будет
давать сильные искажения в оценку.

При оценке элемента кросс-спектра мы усредняли комлекснозначные величины для
отдельных эпох и временных срезов.  Каждая такая величина представляла собой
вектор на плоскости, повернутый на разность фаз между двумя сигналами
относительно оси абсцисс с длиной, пропорциональной произведению амплитуд двух
сигналов.  Такое усреднение приводило к тому, что вектора с постоянной
разностью фаз усиливали друг друга, а вектора со случайной разностью фаз ---
ослабляли. Тем не менее, средний вектор все равно оказывался пропорционален
амплитудам сигналов: мощные источники со случайной фазовой задержкой могли дать
кросс-спектральный коэффициент больше, чем слабые источники со стабильной
разностью фаз.

Для оценки мощности мы можем изменить стратегию при усреднении. Будем усреднять
квадраты длин соответствующих 2-мерных векторов. Тогда полученное среднее
будет давать оценку произведения мощностей взаимодействующих сигналов.
Нормализация элемента кросс-спектра на источниках после применения PSIICOS-проекции
на корень из этой величины будет давать нормализованный коэффициент, ограниченный
по модулю (кросс-спектральный коэффициент --- комплексная величина) отрезком $[0, 1]$.

Похожая стратегия нормализации используется в методе wPLI~\cite{wPLI}.
Эта мера основана на мнимой части кросс-спектра, поэтому информация о мощности
отдельных источников в ней отсутствует. Для нормализации используется
среднее значение модуля мнимой части ``кросс-спектра'', посчитанное для
каждой эпохи отдельно.

\subsection{GO-PSIICOS}

% Описать, как PSIICOS-проекция встраивается в метод IrMxNE.
Globally optimized PSIICOS --- это комбинация метода очистки от
протечки сигнала PSIICOS и метода спарсного решения обратной задачи
IrMxNE.\@

Процедура заключается в следующем.

\begin{enumerate}
    \item Для каждого временного среза рассчитываем векторизованный кросс-спектр 
        в пространстве сенсоров. Усреднение проводится по эпохам.
    \item К каждому временному срезу применяем оператор PSIICOS-проекции. Совокупность 
        спроецированных векторизованных кросс-спектров складываем в матрицу как столбцы. 
        Полученную матрицу будем называть временным рядом кросс-спектра.
    \item В качестве прямого оператора для метода IrMxNE берем матрицу, в которой
        каждый столбец является 2-топографией для двух различных точек коры. Таким
        образом каждой паре точек на коре соответствует столбец полученной матрицы. 
        (для учета свободной ориентации диполя в случае МЭГ --- 4 столбца)
    \item Для полученной матрицы прямой модели и спроецированного
        временного ряда кросс-спектра применяем алгоритм IrMxNE.
\end{enumerate}
