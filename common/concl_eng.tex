%% Согласно ГОСТ Р 7.0.11-2011:
%% 5.3.3 В заключении диссертации излагают итоги выполненного исследования, рекомендации, перспективы дальнейшей разработки темы.
%% 9.2.3 В заключении автореферата диссертации излагают итоги данного исследования, рекомендации и перспективы дальнейшей разработки темы.
\begin{enumerate}
  \item A review of studies of changes in brain functional connectivity in various pathologies was conducted.
  \item A method of cleaning EEG and MEG data from signal leakage was developed, on the basis of which a family of algorithms for phase synchrony estimation was developed, allowing to find networks with close to zero phase delays.
  \item The objective of phase synchrony estimation under conditions of signal leakage was formulated and solved as a problem of optimal filtering.
  \item An algorithm allowing to find networks with close to zero phase delays, optimal in the global sense and allowing to cope with the problem of false positives of the second kind caused by the signal leakage was proposed.
  \item A numerical study of the properties of the proposed projection has been carried out, showing that the developed method allows to suppress the contribution of the subspace of the signal leakage into the matrix of cross-spectral power density estimated on the sensors.
  \item Numerical study of the projection method properties showed that the developed method allows to find networks with close to zero phase delays under conditions of non-invasive MEG measurements, which are characterized by a significant signal leakage between sources
  \item Comparison with currently available algorithms for estimating connectivity  from non-invasive simulation data showed a significant advantage  of the proposed network detection technique in conditions of small phase delays.
  \item The application of the method of signal leakage cleaning to real data allowed to detect physiologically plausible networks, which cannot be detected by other methods.
  \item A numerical study of the influence of the rank values of the proposed projection on the properties of the algorithm was carried out, which made it possible to obtain heuristics for selecting the rank.
  \item Numerical study of the influence of forward model inaccuracies on the solution quality of the proposed algorithm has shown that characteristic for real recordings error ranges in the estimation of forward model have little effect on the quality of the obtained solutions.
  \item A package of utilities in the MATLAB environment was created to accomplish the tasks, which includes tools for generating test data, visualizing the spatial and temporal structure of networks, and finally, software implementations of developed algorithms.
  \item The developments obtained in the course of work on this dissertation were implemented in the Visbrain and Neuropycon software packages available for public use.
\end{enumerate}
