
% Обзор, введение в тему, обозначение места данной работы в
% мировых исследованиях и~т.\:п., можно использовать ссылки на~другие
% работы\ifnumequal{\value{bibliosel}}{1}{~\autocite{Joshi2015}}{}
% % (если их~нет, то~в~автореферате
% автоматически пропадёт раздел <<Список литературы>>). Внимание! Ссылки
% на~другие работы в разделе общей характеристики работы можно
% использовать только при использовании \verb!biblatex! (из-за технических
% ограничений \verb!bibtex8!. Это связано с тем, что одна
% и~та~же~характеристика используются и~в~тексте диссертации, и в
% автореферате. В~последнем, согласно ГОСТ, должен присутствовать список
% работ автора по~теме диссертации, а~\verb!bibtex8! не~умеет выводить в одном
% файле два списка литературы).
% При использовании \verb!biblatex! возможно использование исключительно
% в~автореферате подстрочных ссылок
% для других работ командой \verb!\autocite!, а~также цитирование
% собственных работ командой \verb!\cite!. Для этого в~файле
% \verb!Synopsis/setup.tex! необходимо присвоить положительное значение
% счётчику \verb!\setcounter{usefootcite}{1}!.

% Для генерации содержимого титульного листа автореферата, диссертации
% и~презентации используются данные из файла \verb!common/data.tex!. Если,
% например, вы меняете название диссертации, то оно автоматически
% появится в~итоговых файлах после очередного запуска \LaTeX. Согласно
% ГОСТ 7.0.11-2011 <<5.1.1 Титульный лист является первой страницей
% диссертации, служит источником информации, необходимой для обработки и
% поиска документа>>. Наличие логотипа организации на титульном листе
% упрощает обработку и поиск, для этого разметите логотип вашей
% организации в папке images в формате PDF (лучше найти его в векторном
% варианте, чтобы он хорошо смотрелся при печати) под именем
% \verb!logo.pdf!. Настроить размер изображения с логотипом можно
% в~соответствующих местах файлов \verb!title.tex!  отдельно для
% диссертации и автореферата. Если вам логотип не~нужен, то просто
% удалите файл с логотипом.

{\actuality}
Начиная с зарождения первых цивилизаций и до сегодняшнего момента вопросы
о том, как устроены сознание и разум в человеке и других живых существах, продолжают быть ключевыми для человечества.
Так сложилось, что их осмысление происходило сперва в рамках религиозной,
а затем и философской парадигмы, которые в западноевропейской традиции значительно перекликались и дополняли друг друга.
От платоновского мира идей и парменидовского разграничения истины и мнения о ней через
средневековое понятие об истине как о благодати, исходящей от Бога к tabula rasa эмпириков
и критике чистого разума Канта и языковым играм Витгенштейна и далее эволюционировало
представление о том, как устроено познание человеком окружающей действительности.

Научный подход к вопросам познания сформировался лишь относительно недавно в рамках группы
дисциплин, включающей в себя нейробиологию, нейрофизиологию, когнитивную психологию и
электрофизиологию. В рамках этих дисциплин ключевым для понимания когнитивной функции человека
и животных в самом широком смысле становится устройство центральной нервной системы и, в частности,
головного мозга.

Интересно, что механизмы познания связывались с функцией головного мозга не всегда даже
в рамках материалистического описания. Так например, Аристотель считал источником мысли сердце,
а мозгу отводил лишь роль радиатора, охлаждающего кровь. Тем не менее, уже в эпоху классической
античности Гален сформировал идею о том, что именно мозг является источником мысли,
а значит и тем инструментом, с помощью которого реализуется познание.
На укоренение идеи о том, что изучение работы центральной нервной системы и её высшего отдела
--- коры больших полушарий --- способно предложить ответ на фундаментальный вопрос
“что представляет собой человеческий интеллект”, ушло еще более полутора тысяч лет.
На сегодняшний день удовлетворительного ответа на этот вопрос по-прежнему нет и появится он,
вероятно, не скоро. Однако в ходе  долгого и непростого движения к этой Ultima Thule
наше понимание более прикладных вещей, вращающихся около нейрофизиологии, несравненно обогатилось.
С практической точки зрения трудно переоценить значение понимания работы ЦНС для медицины, не ограничиваясь, однако, лишь ею.

Сегодня, вместе со всеобщим размыванием междисциплинарных границ, нейронауки все больше оказываются
связанными с более техническими и инженерно-прикладными дисциплинами.
Так, в 1943 году вдохновленные архитектурой нейронных ансамблей живого мозга Маккаллок и Питтс
создают первую вычислительную модель нейронной сети, породив тем самым столь популярный сегодня
класс алгоритмов машинного обучения. Все большую популярность приобретают сегодня мозг-компьютерные интерфейсы,
позволяющие формировать управляющую команду на основе электромагнитной активности мозга напрямую,
что открывает совершенно новые перспективы для интеграции человека с машиной.

В этой связи развитие методов, связанных с изучением строения и работы мозга а также декодирование
порождаемых им сигналов представляет сегодня чрезвычайный интерес.
Вместе с тем, за последние сто лет благодаря резкому скачку в развитии электроники,
физики и компьютерных наук набор инструментов в руках ученого-нейрофизиолога существенно обогатился.
На сегодняшний день существующие методы с точки зрения необходимости хирургического вмешательства
для проведения измерений можно разделить на инвазивные и неинвазивные.

К первой группе относится интракраниальная энцефалография --- метод,
в котором электрические потенциалы записываются напрямую с коры больших полушарий.
Недостатки и преимущества такого подхода очевидны. К первым прежде всего относится необходимость
хирургического вмешательства для проведения измерений, что существенно ограничивает возможности
исследователя-нейрофизиолога в получении данных для исследования.
На практике осуществление таких измерений на человеке возможно лишь для пациентов,
прошедших операцию на мозге в связи с каким-либо неврологическим заболеванием, как правило эпилепсией.
В ходе операции для мониторинга активности мозга после хирургического вмешательства на кору головного
мозга устанавливаются электроды, регистрирующие электрическую активность.
Ясно, что количество таких данных, как и  возможность проведения каких-либо сложных
когнитивных экспериментов на пациентах, прошедших операцию на мозге, весьма ограничены.
При этом качество электрического сигнала, записанного в непосредственной близости от его источника,
несравненно выше того, что можно получить, записывая электроэнцефалограмму с поверхности кожи головы.

Неинвазивные методы, с другой стороны, представляют собой намного более гибкий инструмент
для исследований головного мозга человека в силу отсутствия необходимости проведения операции.
Для изучения анатомической организации мозга а также в качестве вспомогательного инструмента
при анализе активности нейронных популяций коры используются методы структурной нейровизуализации,
такие как магнитно-резонансная томография (МРТ),
компьютерная томография (КТ) и диффузионная тензорная визуализация (ДТВ).
Они позволяют неинвазивно получать статические трехмерные изображения тканей головного мозга.
Для изучения динамической активности нейронов используются функциональные методы нейровизуализации,
а именно ---  функциональная магнитно-резонансная томография (фМРТ),
позитронно-эмиссионная томография (ПЭТ), электроэнцефалогафия (ЭЭГ), а также магнитная энцефалография (МЭГ).

При этом лишь последние два метода измеряют электрическую активность мозга непосредственно,
тогда как фМРТ и ПЭТ меряют локальный кровоток, который меняется сравнительно медленно,
существенно ограничивая временное разрешение этих методов.
Так, для ЭЭГ и МЭГ временное разрешение оказывается равным $\approx$ 1мс, а методы,
измеряющие локальный кровоток, позволяют разрешить лишь процессы с характерными временами
порядка одной секунды и медленнее. Вместе с тем, осцилляторные электрофизиологические процессы, порождаемые
тканями головного мозга, имеют характерные времена от 0.1 секунды и быстрее \ifnumequal{\value{bibliosel}}{1}{~\autocite{buzsaki}}{}.
Таким образом, среди всех имеющихся на сегодняшний день инструментов анализа, \emph{только ЭЭГ и МЭГ
позволяют осуществлять неинвазивные записи сравнительно быстрой электрофизиологической
активности головного мозга}, что делает их незаменимым инструментом при изучении \emph{осцилляций}
их \emph{синхронизации} в головном мозге человека.

Способность порождать осцилляции или ритмическую токовую активность является существенной чертой,
присущей работе нейронных популяций. Природа возникающих ритмов, а также их функциональное назначение
на сегодняшний день остаются предметом изучения, и нет единой, принятой всеми точки зрения на этот счет.
Однако, широко принимается гипотеза, согласно которой осцилляции, порождаемые различными нейронными популяциями,
служат механизмом, позволяющим различным функционально-специфичным областям мозга
избирательно осуществлять обмен информацией друг с другом. Иными словами, предполагается, что осцилляции ответственны за
процессы \emph{функциональной интеграции} \cite{}.

Согласно существующим представлениям, функциональная интеграция нейронных ансамблей осуществляется за
счет синхронизации порождаемых этими ансамблями осцилляций. При этом области коры, в которых ритмическая
активность синхронизована, получают возможность эффективнее передавать информацию, а десинхронизованные
области, напротив, перестают обмениваться сигналами. Такое представления об организации эффективных каналов
передачи информации между нейронными ансамблями за счет синхронизации получило в литературе название
<<взаимодействие через когерентность>> (в английском варианте communication through coherence, CTC)
\ifnumequal{\value{bibliosel}}{1}{~\autocite{Fries2015}}{}. Иными словами, синхронизация осцилляций
являются тем механизмом, который позволяет динамически связывать в сети функционально специфичные области мозга
для выполнения определенной когнитивной задачи. Изучение таких сетей, возникающих и распадающихся в процессе решения мозгом
определенных когнитивных задач, является сегодня одной из центральных тем в изучении мозговой активности, как в норме,
так и при патологии \ifnumequal{\value{bibliosel}}{1}{~\autocite{varela}}{},
\ifnumequal{\value{bibliosel}}{1}{~\autocite{baker}}{}, \ifnumequal{\value{bibliosel}}{1}{~\autocite{ossadtchi}}{},
\ifnumequal{\value{bibliosel}}{1}{~\autocite{Bastin2017}}{}, \ifnumequal{\value{bibliosel}}{1}{~\autocite{ossadtchi}}{}, \cite{myself}, \cite{myself}.
С точки зрения исследования таких сетей выделяют понятие \emph{функциональной коннективности}, понимая
под этим статистические закономерности в одновременной активации (в самом широком смысле) различных областей мозга \cite{}.
При этом вывод о том, что эти области мозга работали синхронно, делается на основании вычисления
определенной метрики, отражающей степень сходства измеренных (или математически восстановленных) в этих областях сигналов.
Такие метрики называются \emph{мерами коннективности}.

Многое в области изучения функциональной коннективности было сделано с использованием технологии фМРТ,
однако отмеченное выше ограничение фМРТ в виде крайне плохого временного разрешения делает электрофизиологические
методы измерений незаменимыми при анализе коннективности. Особое место при этом занимает магнитная энцефалография,
которая в сочетании с методами восстановления сигнала на коре головного мозга
в силу более высокой точности прямой модели по сравнению с ЭЭГ предоставляет
в руки исследователя уникальное сочетание менее чем сантиметрового разрешения по пространству
и миллисекундного разрешения по времени \ifnumequal{\value{bibliosel}}{1}{~\autocite{hamalainen}}{},
\ifnumequal{\value{bibliosel}}{1}{~\autocite{Baillet}}{}, \ifnumequal{\value{bibliosel}}{1}{~\autocite{Gross2013}}{}.


\subsubsection*{Существующие методики анализа синхронностей}

Вообще, оценка коннективности на основании неинвазивных электрофизиологических данных,
 представляет собой сложную инженерную задачу,
на решение которой научное сообщество уже потратило немало сил.
За последние несколько десятилетий было разработано и опробовано множество методов
оценки функциональной коннективности от стандартных подходов,
включающих меры синхронизации сигналов во временной и
частотной области (таких как корреляция и когеренция), до более изощренных,
зачастую нелинейных мер коннективности
\ifnumequal{\value{bibliosel}}{1}{~\autocite{Marzetti2008}}{},\ifnumequal{\value{bibliosel}}{1}{~\autocite{Schoffelen2009}}{},
\ifnumequal{\value{bibliosel}}{1}{~\autocite{Colclough2015}}{},
\ifnumequal{\value{bibliosel}}{1}{~\autocite{kaminski}}{},\ifnumequal{\value{bibliosel}}{1}{~\autocite{greenblatt_conn}}{},
\ifnumequal{\value{bibliosel}}{1}{~\autocite{hillebrand}}{},\ifnumequal{\value{bibliosel}}{1}{~\autocite{imcoh}}{},
\ifnumequal{\value{bibliosel}}{1}{~\autocite{Lachaux1999}}{},\ifnumequal{\value{bibliosel}}{1}{~\autocite{env_corr}}{},
\ifnumequal{\value{bibliosel}}{1}{~\autocite{Brookes2012}}{},\ifnumequal{\value{bibliosel}}{1}{~\autocite{Brookes2011}}{}
\ifnumequal{\value{bibliosel}}{1}{~\autocite{Hillebrand2012}}{},\ifnumequal{\value{bibliosel}}{1}{~\autocite{Hipp2012}}{},
\ifnumequal{\value{bibliosel}}{1}{~\autocite{PLI}}{},\ifnumequal{\value{bibliosel}}{1}{~\autocite{wPLI}}{},
\ifnumequal{\value{bibliosel}}{1}{~\autocite{Chella2015}}{},\ifnumequal{\value{bibliosel}}{1}{~\autocite{Chella2016}}{},
\ifnumequal{\value{bibliosel}}{1}{~\autocite{Wibral2011}}{},\ifnumequal{\value{bibliosel}}{1}{~\autocite{Chella2016}}{}.
\ifnumequal{\value{bibliosel}}{1}{~\autocite{Ioannides2000}}{}.
Ни одна из предложенных мер, обладая своими достоинствами и недостатками, не является, однако, универсальной в силу
сохраняющихся технических затруднений \ifnumequal{\value{bibliosel}}{1}{~\autocite{Colclough2016}}{},\ifnumequal{\value{bibliosel}}{1}{~\autocite{Bastos2016}}{}.

Одной из наиболее существенных проблем, возникающих при оценке функциональной коннективности является
так называемая \emph{протечка сигнала}, объясняемая тем, что обратная задача для ЭЭГ/МЭГ
плохо поставлена. Практически это означает, что имея ограниченный набор измерений нельзя однозначно восстановить
конфигурацию источников, породивших сигнал. Из этого, в свою очередь, следует невозможность полностью
размешать сигналы, записанные сенсорами, --- в каждый из восстановленных сигналов неизбежно будут
подмешаны сигналы от остальных источников. Следовательно, все восстановленные сигналы будут в какой-то
степени похожи друг на друга, даже если исходные сигналы не демонстрировали никаких признаков синхронизации, а значит и
меры коннективности, будучи мерами сходства сигналов, будут демонстрировать завышенные значения
(для более формального изложения проблемы протечки сигнала см. главу 1). Возникает проблема различения истинной синхронности
и той, которая порождена фундаментальными ограничениями неинвазивной электрофизиологии.

Впервые попытка решения этой проблемы была предпринята в 2004 году в статье Г. Нолте
\ifnumequal{\value{bibliosel}}{1}{~\autocite{imcoh}}{},
в которой авторы предлагают использовать в качестве меры коннективности величину, называемую мнимой частью когеренции.
Для этого каждый сигнал сначала необходимо перевести в частотную область,
затем для каждой пары сигналов посчитать функцию когерентности, и наконец, взять от полученной величины её мнимую часть.
Идея такого метода оценки коннективности заключается в том, что мнимая часть когеренции имеет ненулевое значение лишь
для сигналов с ненулевой разностью фаз, тогда как эффект протечки сигнала всегда проявляется в виде ложной синхронизации
с нулевой фазовой задержкой, давая тем самым вклад лишь в действительную часть когеренции. Действительно, такой подход
существенно повышает устойчивость метода к протечке сигнала. Тем не менее, так как функция когерентности
нормируется на оцененные мощности сигналов (которые, будучи чисто действительными величинами, подвержены влиянию
протечки сигнала) итоговые оценки коннективности по мнимой части когеренции также, пусть и в меньшей степени,
испорчены эффектом протечки.

На эту деталь в 2007 году обратил внимание Стэм в своей статье
\ifnumequal{\value{bibliosel}}{1}{~\autocite{PLI}}{}. Стэм предложил использовать для оценки синхронизации
вместо мнимой части когеренции среднее значение знака разности фаз двух сигналов.
Такая мера оказывается очень похожей на мнимую часть когеренции, однако нормировка
(скрытая в операции взятия знака мнимой части) теперь производится лишь на чисто мнимые величины, которые не
зависят от протечки сигнала. Стэм назвал свою меру индексом фазовой задержки (phase lag index, PLI).

Следующей ступенью эволюции в цепочке методов, основанных на идее мнимой части когеренции стала
мера, называемая взвешенным индексом фазовой задержки (weighted phase lag index, wPLI). Ее описал Винк с
соавторами в своей статье 2011 года. Мотивацией к разработке новой меры коннективности послужил тот
факт, что мера PLI оказалась слишком неустойчивой по отношению к шуму. Основной недостаток индекса фазовой
задержки, как и его преимущество перед мнимой частью когеренции, кроется в операции взятия знака.
Дело в том, что для шумовых источников случайно меняющийся знак разности фаз оказывает слишком большое
влияние на измерения. Чтобы избавиться от этого недостатка, Винк предложил взвешивать знак разности фаз
на амплитуду мнимой части соответствующих кросс-спектральных коэффициентов. Таким образом, вклад от шумовых
источников малой амплитуды оказывается малым, что делает меру более устойчивой.

Семейство мер коннективности, основанных на мнимой части когеренции, не исчерпывается обозначенными
тремя подходами. Аналогичная идея, но под немного другим углом, была применена в статье
\ifnumequal{\value{bibliosel}}{1}{~\autocite{Hipp2012}}{} 2012 года. В ней в качестве меры синхронности
авторы используют корреляцию огибающих двух узкополосных сигналов. Проблема протечки сигнала
в статье решена следующим образом:
на коре восстанавливаются два временных ряда, затем один из них проецируется ортогонально второму,
после чего вычисляются огибающие и рассчитывается коэффициент корреляции между ними.
Такой подход, основанный на ортогонализации временных рядов, оказывается эквивалентным взятию
мнимой части  соответствующего кросс-спектрального коэффициента.


% Еще можно написать про wedge music

Все изложенные методики, основанные на мнимой части когеренции, имеют,
однако, один существенный недостаток, а именно --- все они не чувствительны к
синхронизации с нулевой фазовой задержкой. Как уже отмечалось выше, операция
взятия мнимой части когеренции эквивалентна удалению из данных профилей синхронизации
с нулевой фазовой задержкой. Практически это означает не только невозможность детектирования
сетей, синхронизированных с нулевой фазой, но и плохое отношение сигнал / шум (ОСШ) для
сетей, для которых фазовая задержка малая. Более того, чем ближе эта фазовая задержка к нулю,
тем ОСШ хуже, и наоборот, чем разность фаз двух сигналов ближе к $\pi / 2$, тем ОСШ выше.

Ясно, что такое неравномерное распределение детекторных характеристик метода по
фазовым задержкам крайне ограничивает возможности исследователя, тем более что
синхронизация с нулевой фазой, по всей видимости, является важным и широко представленным
явлением в организации осцилляторной мозговой активности \cite{}, \cite{}.

По этой причине на сегодняшний день в неинвазивной электрофизиологии имеется острая
потребность в появлении иснструмента измерения коннективности, который, с одной стороны,
будет устойчив к эффекту протечки сигнала, а с другой --- будет способен обнаруживать сети
для всего спектра фазовых задержек.


Попытка создать такой метод была предпринята в 2015 году в статье
\ifnumequal{\value{bibliosel}}{1}{~\autocite{Wens2015}}{}
В ней авторы использовали принципиально иной метод борьбы с эффектом протечки сигнала.
Идея этого метода состоит в использовании информации
о взаимном расположении источников сигнала и сенсоров для конструкции особых пространственных
фильтров, которые позволяют очистить один источник
от сигнала, пришедшего от другого источника для последующего измерения
некоторого индекса синхронности. Авторы в качестве такого индекса предложили использовать корреляцию огибающих сигналов. Более детально структура
предложенного метода такова. Во-первых, по сигналам на сенсорах восстанавливаются сигналы на источниках.
Далее фиксируется один из источников на коре.
Все остальные источники пространственно фильтруются от активности, протекшей от фиксированного источника.
Далее меряется корреляция огибающих между фиксированным источником и всеми остальными.
Чтобы получить значение коннективности для каждой пары источников нужно повторить процедуру, выбирая
в качестве фиксированного источника каждый из оставшихся.
Наконец, так как полученная матрица коннективностей будет вообще говоря асимметричной, значения
значения коннективности для пар $(i,j)$ и $(j,i)$ усредняются. Такую эвристику авторы статьи назвали
методом геометрической поправки (geometric correction scheme, GCS).

Метод GCS концептуально явился серьезным продвижением вперед, так как теперь
появилась возможность детектировать сети малыми сдвигами фаз
оставаясь (по крайней мере, в теории) вне влияния эффекта протечки сигнала.
В действительности, однако, такой метод коррекции лишь частично нивелирует этот эффект,
так как он не учитывает протечку от третьих источников при оценке коннективности. В качестве примера
можно рассмотреть ситуацию, когда имеется три мощных источника,
никакие два из которых не были синхронизировани.
В такой постановке несмотря на отсутствие синхронностей метод GCS будет давать высокие значения коннективности
для всех трех пар связей, так как хотя для каждой пары коррекция очистит сигналы от протечки друг в друга,
сигнал от третего источника, протекая в каждый источник из пары, создаст общую компоненту в восстановленных источниках.
В результате коннективность, которую мы измерим для исходно не синхронных
источников, после геометрической коррекции для пары источников фактически будет отражать степень протечки
от третьего источника в каждый сигнал из пары. Ясно, что если третий сигнал лежит близко к первым двум,
эффект протечки будет весьма существенным. В результате для большого количества активных источников
даже очищенный сигнал оказывается крайне загрязненным, что существенно ограничивает применимость GCS к
практическим задачам.



Таким образом, \emph{до сих пор не существует метода оценки коннективностей,
позволяющего надежно детектировать сети с малыми фазовыми задержками и при этом свободного
от эффекта протечки сигнала}


\subsubsection*{Использование априорной информации в оценке коннективностей}

Современная практика использования мер коннективности в нейрофизиологических
исследованиях в подавляющем большинстве случаев следует одной из двух возможных схем.
Первая схема предполагает изучение нейрофизиологического эффекта  в \emph{пространстве сенсоров},
т.е. выбранная исследователем мера коннективности применяется непосредственно к сигналам,
записанным электродами.
Второй вариант предполагает переход в пространство источников --- сначала оцениваются
возможные источники записанной электрофизиологической активности на коре, а затем к этим
источникам применяется та или иная  мера коннективности.

Очевидным образом, первый вариант позволяет дать лишь весьма грубую оценку локализации
узлов восстановленных сетей, поэтому часто используется лишь как первое приближение к результату.
Более интересным, хотя и более сложным концептуально и более
трудоемким с точки зрения вычислительных ресурсов, является второй вариант, в котором
сначала оценивается сигнал на источниках, а потом считается мера коннективности.

Так как оценка источников в неинвазивной электрофизиологии является
плохо поставленной обратной задачей \ifnumequal{\value{bibliosel}}{1}{~\autocite{hamalainen}}{},
ее решение не определено однозначно. Иными словами, любые электрофизиологические измерения,
сделанные ограниченным набором сенсоров, можно объяснить бесконечным количеством конфигураций
источников электромагнитной активности, расположенных на коре. Разумеется, часть конфигураций
или решений обратной задачи, будет нефизиологична. Следовательно, среди бесконечного набора
решений необходимо выбрать то, которое с одной стороны хорошо объясняет наблюдения, а с другой ---
соответствует имеющимся представлениям о физиологии мозга.

Таким образом, решение обратной задачи в неинвазивной электрофизиологии всегда требует внедрения
в модель дополнительной априорной информации о структуре решения. Не в каждом методе
решения обратной задачи можно явно указать тот момент, в который делается дополнительное
предположение о структуре решения, однако большая часть таких методов
(например, \cite{mne}, \cite{min_current}, \cite{loreta})
может быть описана в терминах Тихоновской регуляризации \cite{tikhonov}, позволяющей свести задачу поиска
решения к минимизации функционала, состоящего из двух членов: первый --- насколько хорошо
решение объясняет измеренный сигнал, второй --- насколько оно соответствует тому классу решений,
который мы считаем <<физиологичным>>. При этом, формализация понятия <<физиологичный>> может
включать в себя широкий спектр различных предположений о структуре решения
--- от естественного требования непрерывности по пространству и времени (как в MNE, \cite{mne}) до
информации об анатомическом строении мозга.

Оценка источников в такой постановке происходит оптимально с точки зрения минимизации
выбранного функционала, однако в задаче оценки коннективностей оценка источников не
является самоцелью, и их оптимальная оценка не гарантирует оптимальной оценки
достаточных статистик синхронности в пространстве источников

Таким образом, двухступенчатая процедура оценки коннективностей, вообще говоря,
дает субоптимальные результаты с точки зрения оценивания соответствующих статистик.
% Ситуация здесь может быть улучшена рассмотрением порождающих моделей для этих статистик статистик.
% В данной работе такой подход изложен в применении к матрице \emph{кросс-спектральной плотности},
% являющейся достаточной статистикой для оценки фазовой синхронности в
% пространстве источников \cite{cross_sufficient}.
Оптимальное оценивание по наблюдениям статистик синхронности в пространстве источников, требует 
рассмотрения порождающих моделей с формулировкой априорных посылок для сетей вместо таковых для
источников. В частности, желательно было бы находить такие решения обратной задачи, которые
объясняют измерения \emph{минимальным набором сетей}. Мотивация такого подхода кроется в известном
принципе бритвы Оккама --- объяснение наблюдаемых данных должно быть максимально простым.
Известно, что решения такого вида, то есть те, в которых число отдельных структурных элементов,
объясняющих данные, минимально, реализуются при помощи спарсной регуляризации.
Как вводить такую регуляризацию в рамках двухступенчатой процедуры, однако, не совсем
понятно.



 % \ifsynopsis
% Этот абзац появляется только в~автореферате.
% Для формирования блоков, которые будут обрабатываться только в~автореферате,
% заведена проверка условия \verb!\!\verb!ifsynopsis!.
% Значение условия задаётся в~основном файле документа (\verb!synopsis.tex! для
% автореферата).
% \else
% Этот абзац появляется только в~диссертации.
% Через проверку условия \verb!\!\verb!ifsynopsis!, задаваемого в~основном файле
% документа (\verb!dissertation.tex! для диссертации), можно сделать новую
% команду, обеспечивающую появление цитаты в~диссертации, но~не~в~автореферате.
% \fi

% {\progress}
% Этот раздел должен быть отдельным структурным элементом по
% ГОСТ, но он, как правило, включается в описание актуальности
% темы. Нужен он отдельным структурным элементом или нет ---
% смотрите другие диссертации вашего совета, скорее всего не нужен.
Имея в виду все вышесказанное, можно заключить, что на сегодняшний день
процедура оценки коннективности по неинвазивным электрофизиологическим данным с одной стороны
все еще является плохо разработанной и нуждается в улучшениях (неслучайна
регулярная публикация новых методологический статей по теме оценки коннективностей),
а с другой является ключевым инструментом для современной нейрофизиологии,
следуя за смещением акцента в изучении мозга от активации его отдельных областей к взаимодействию 
между ними.

{\aim} данной работы, таким образом, является разработка метода
оценки коннективностей, который
\begin{itemize}
        \item позволяет оценить фазовую синхронность в условиях взаимной протечки сигналов
        \item чувствителен к сетям с малыми фазовыми задержками
        \item оптимален с точки зрения оценки достаточной статистики для коннективности
        \item способен учитывать априорную информацию об организации фазовых синхронностей
\end{itemize}
а также его валидация в применении к симуляционным МЭГ-данным.

Для~достижения поставленной цели необходимо было решить следующие {\tasks}:
\begin{enumerate}
  \item Разработать
  \item Исследовать, разработать, вычислить и~т.\:д. и~т.\:п.
  \item Исследовать, разработать, вычислить и~т.\:д. и~т.\:п.
\end{enumerate}


{\novelty}
\begin{enumerate}
  \item Впервые \ldots
  \item Впервые \ldots
  \item Было выполнено оригинальное исследование \ldots
\end{enumerate}

{\influence} \ldots

{\methods} \ldots

{\defpositions}
\begin{enumerate}
  \item Первое положение
  \item Второе положение
  \item Третье положение
  \item Четвертое положение
\end{enumerate}
В папке Documents можно ознакомиться в решением совета из Томского ГУ
в~файле \verb+Def_positions.pdf+, где обоснованно даются рекомендации
по~формулировкам защищаемых положений.

{\reliability} полученных результатов обеспечивается \ldots \ Результаты находятся в соответствии с результатами, полученными другими авторами.


{\probation}
Основные результаты работы докладывались~на:
перечисление основных конференций, симпозиумов и~т.\:п.

{\contribution} Автор принимал активное участие \ldots

%\publications\ Основные результаты по теме диссертации изложены в ХХ печатных изданиях~\cite{Sokolov,Gaidaenko,Lermontov,Management},
%Х из которых изданы в журналах, рекомендованных ВАК~\cite{Sokolov,Gaidaenko},
%ХХ --- в тезисах докладов~\cite{Lermontov,Management}.

\ifnumequal{\value{bibliosel}}{0}{% Встроенная реализация с загрузкой файла через движок bibtex8
    \publications\ Основные результаты по теме диссертации изложены в XX печатных изданиях,
    X из которых изданы в журналах, рекомендованных ВАК,
    X "--- в тезисах докладов.%
}{% Реализация пакетом biblatex через движок biber
%Сделана отдельная секция, чтобы не отображались в списке цитированных материалов
    \begin{refsection}[vak,papers,conf]% Подсчет и нумерация авторских работ. Засчитываются только те, которые были прописаны внутри \nocite{}.
        %Чтобы сменить порядок разделов в сгрупированном списке литературы необходимо перетасовать следующие три строчки, а также команды в разделе \newcommand*{\insertbiblioauthorgrouped} в файле biblio/biblatex.tex
        \printbibliography[heading=countauthorvak, env=countauthorvak, keyword=biblioauthorvak, section=1]%
        \printbibliography[heading=countauthorconf, env=countauthorconf, keyword=biblioauthorconf, section=1]%
        \printbibliography[heading=countauthornotvak, env=countauthornotvak, keyword=biblioauthornotvak, section=1]%
        \printbibliography[heading=countauthor, env=countauthor, keyword=biblioauthor, section=1]%
        \nocite{%Порядок перечисления в этом блоке определяет порядок вывода в списке публикаций автора
                vakbib1,vakbib2,%
                confbib1,confbib2,%
                bib1,bib2,%
        }%
        \publications\ Основные результаты по теме диссертации изложены в~\arabic{citeauthor}~печатных изданиях,
        \arabic{citeauthorvak} из которых изданы в журналах, рекомендованных ВАК,
        \arabic{citeauthorconf} "--- в~тезисах докладов.
    \end{refsection}
    \begin{refsection}[vak,papers,conf]%Блок, позволяющий отобрать из всех работ автора наиболее значимые, и только их вывести в автореферате, но считать в блоке выше общее число работ
        \printbibliography[heading=countauthorvak, env=countauthorvak, keyword=biblioauthorvak, section=2]%
        \printbibliography[heading=countauthornotvak, env=countauthornotvak, keyword=biblioauthornotvak, section=2]%
        \printbibliography[heading=countauthorconf, env=countauthorconf, keyword=biblioauthorconf, section=2]%
        \printbibliography[heading=countauthor, env=countauthor, keyword=biblioauthor, section=2]%
        \nocite{vakbib2}%vak
        \nocite{bib1}%notvak
        \nocite{confbib1}%conf
    \end{refsection}
}
При использовании пакета \verb!biblatex! для автоматического подсчёта
количества публикаций автора по теме диссертации, необходимо
их~здесь перечислить с использованием команды \verb!\nocite!.
