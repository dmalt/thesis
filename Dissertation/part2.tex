\chapter{Решение обратной задачи в пространстве матриц кросс-спектральной плотности} \label{chapt2}

% План.
% Методы решения обратной задачи для МЭГ/ЭЭГ.
% Сканирующий подход. Нахождение ориентации диполя.
% MUSIC, RAP-MUSIC, бимформеры. Сканирующие методы решения обратной задачи для анализа коннективностей.
% Алгоритм DICS. Модификация DICS для оценки imcoh в пр-ве источников. Wedge MUSIC.
% Методы глолбальной оптимизации.
% Оценка активностей при помощи псевдообратной матрицы.
% Априорная информация о структуре решения. MNE, MinimumCurrent, mxNormMNE.
% Адаптация MxNormMNE к решению обратной задачи для поиска синхронностей.
% Техники ускорения расчета и оптимизации вычислительных ресурсов.
% Active set и связь с MUSIC.
% Критерий останова алгоритма.
% Сокращение размерности пр-ва сенсоров.

\section{Введение}

В предыдущей главе нами была получена система уравнений, связывающая
коэффициенты матрицы кросс-спектральной плотности мощности в пространстве
источников с аналогичными коэффициентами в пространстве сенсоров. Хотя это
уравнение и является линейными относительно неизвестных величин $c_{ij}^{ss}$,
процедура нахождения решения, которое адекватно описывало бы поведение реальных
систем, осложнено тем фактом, что рассматриваемая система уравнений существенно
недоопределена, а следовательно для нее существует бесконечное количество
решений, далеко не все из которых осуществимы для реальных систем
взаимодействующих нейронных ансамблей.

Вообще, система уравнений~\ref{eq:cp_final_re_im} для фиксированных ориентаций
диполей по своей структуре ничем кроме шумового слагаемого не отличается от
систем~\ref{eq:BV_generative_matrix}.
Система уравнений~\ref{eq:BV_generative_matrix} на неизвестные величины $\mathcal{Q}$ также
является линейной и недоопределенной. Некоторые отличия возникают лишь при
рассмотрении свободно ориентированных диполей.

Задачу оценки величин, положений и ориентаций токовых диполей $\mathcal{Q}$ на
основании измерений $\mathcal{B}, \mathcal{V}$ в электрофизиологии принятно
называть \emph{обратной задачей} МЭГ/ЭЭГ.  Как и в случае оценки
кросс-спектральных коэффициентов, решение обратной задачи МЭГ/ЭЭГ не
единственно.  Для выбора какого-либо одного решения $\mathcal{Q}$ по коре
применяют различные эвристики, ограничивающие выбор из бесконечного множества
возможных вариантов.  Как правило, получающееся решение отвечает тому или иному
критерию оптимальности в соответствии с используемой эвристикой, при условии,
что предположения модели выполняются.  Во введении к этой главе мы рассмотрим
основные методы решения обратной задачи для поиска активных токовых диполей на
коре, а затем перейдем к рассмотрению методик для оценки кросс-спектральных
коэффициентов с учетом свободной ориентации.

Все существующие методики решения обратной задачи МЭГ/ЭЭГ можно условно
разделить на два класса.
К первому классу относятся алгоритмы, основанные на
поиске заранее заданного числа эквивалентных токовых диполей, объясняющих измерения
наилучшим образом. При этом оцениваются положения и ориентации этих диполей.
К этому классу относится необходимый нам для дальнейшего изложения алгоритм % Так ли это?
MUSIC (multiple signal classification), а также его модификация RAP-MUSIC
(recursively applied and projected MUSIC), которые условно можно назвать сканирующими.

Второй класс алгоритмов, называемых в литературе <<имиджинговыми>>, ставит задачу
отыскания активности, распределенной сразу по всей коре. Ко второй группе методов
можно условно отнести методы оптимальной пространственной фильтрации, которые
восстанавливают сигнал отдельно для каждой выбранной точки коры в соответствии
с неким локальным критерием оптимальности, а также методы, основанные на выборе
решения с минимальной нормой.

Суть подхода оптимальной фильтрации состоит в том, что для
фиксированной точки внутри объема мозга ставится задача нахождения
пространственного фильтра, оптимизирующего определенную характеристику сигнала,
восстанавливаемого при помощи этого фильтра. В качестве такой характеристики
может выступать, например, отношение сигнал/шум, или же мы можем
руководствоваться критерием минимизации протечки сигнала от других источников в
точку, в которой мы хотим восстановить активность.  Здесь важно отметить, что
конкретный вид решения, полученного в результате оптимизации выбранного
функционала качества будет зависить также от предполагаемой пространственной
структуры шума.

Отметим, что структура восстановленной после применения совокупности найденных
фильтров активации на коре при таком подходе, вообще говоря, субоптимальна с
точки зрения объяснения сигнала, измеренного сенсорами (так как мы
оптимизировали другой функционал качества). Проблема недоопределенности
системы уравнений при этом в некотором смысле остается за скобками, так как для
каждой точки коры решение восстанавливается индивидуально --- без учета вклада
в решение активаций, восстановленных в других точках коры.  Таким образом, для
алгоритмов оптимальной фильтрации найденное решение является
оптимальным в локальном, но не в глобальном смысле.

Задача отыскания активаций, наилучшим образом объясняющих измерения, (т.е.
оптимальных в глобальном смысле) ставится для другого подкласса имиджинговых
методов.
При этом, как уже было отмечено выше, в силу
недоопределенности системы линейных уравнений, связывающих активации на коре с
сигналом на сенсорах, существует бесконечное множество конфигураций источников,
идеально объясняющих померенный сигнал. Тем не менее, среди таких решений в
силу зашумленности измерений а также неточностей при построении прямой модели
реальное распределение активаций (в выделенных точках) коры не содержится, так
как эти <<идеальные>> с точки зрения объяснения измерений решения объясняют в том
числе и шумовую компоненту, которая зачастую оказывается больше или сравнима по
амплитуде с истинной активацией.

Итак, при решении обратной задачи методами глобальной оптимизации существует
две проблемы: бесконечное множество возможных решений и зашумленность
измерений. Чтобы справиться с первой проблемой, для выбора из бесконечного
множества решений пользуются критерием минимальности нормы решения. Иными
словами, среди всех возможных конфигураций первичных токов в объеме (или на
поверхности коры) мозга в качестве решения выбирается такая конфигурация, норма
которой минимальна среди всех возможных.  Условие минимальности нормы в
некотором смысле является следствие принципа бритвы Оккама: мы ищем наиболее
простое решение, удовлетворяющее наблюдениям. Какие решения при этом считать
простыми"--- неочевидный вопрос.  Ответ на него зависит от выбора конкретного
вида нормы, которую мы хотим минимизировать.  Наиболее популярными вариантами
являются $L_2$- и $L_1$-нормы.  Наиболее простой пример минимальной $L_2$-нормы
решения соответствует случаю, когда проблему зашумленности данных мы оставляем
без внимания.  Тогда решение, соответствующее минимуму $L_2$-нормы, получается
применением оператора, соответствующего псевдообратной матрице, взятой для
матрицы прямой модели.  Для $L_1$-нормы ситуация несколько сложнее, так как
решение не может быть получено в явном виде, и требуется численная оптимизация
соответствующего функционала, сводящаяся к задаче линейного программирования.

Рассмотрим теперь, каким образом решается проблема зашумленности данных. Здесь
вновь существует два подхода.  Первый из них используется значительно реже и
состоит в удалении шумовой компоненты из данных посредством сокращенного
сингулярного разложения матрицы прямой модели. Такой подход, например,
использовали авторы, метода Minimum Current Estimate (MCE)~\ref{mce},
порождающего решения с минимальной $L_1$-нормой.

Другой, более популярный подход состоит в использовании тихоновской
регуляризации. В рамках этого подхода норма решения и глобальная ошибка в
объяснении измерений минимизируются совместно, как части одного общего
функционала качества, позволяя тем самым соблюсти баланс между простотой
решения и тем, насколько хорошо оно объясняет измерения, (в том числе,
содержащийся в них шум). Соотношение между <<простотой>> решения и величиной ошибки при
таком подходе можно регулировать настраивая величину метапараметра,
называемого параметром регуляризации. Меняя значение параметра регуляризации
мы стремимся найти такое значение, при котором полученное решение объясняет
только <<полезную>> часть сигнала, записанного сенсорами и полностью игнорирует
шумовую компоненту.

Отметим, что с точки зрения байесовской статистики тихоновская регуляризация
эквивалентна нахождению такого решения обратной задачи, которое соответствует
точке максимума апостериорной плотности вероятности. Минимизируемая норма решения в такой
интерпретации задается априорной плотностью распределения вероятности.

Среди методов, основанных на тихоновской регуляризации, отметим прежде всего
Minimum Norm Estimate (MNE) \ref{mne}, минимизирующий $L_2$-норму решения,
и его вариацию"--- метод dSPM, нормирующий величину восстановленного значения первичного
тока в каждой точке на оцененную величину шума в ней же.

\subsection{Сканирующие алгоритмы} \label{sect_dics}


Наиболее естественным алгоритма поиска фиксированного числа эквивалентных
токовых диполей (dipole fitting), объясняющих данные наилучшим способом,
является оптимизация их положений и ориентаций методом наименьших квадратов.
Такой подход, однако, обладает существенными недостатками, к которым относится
невыпуклость целевой функции при такой оптимизации,
что приводит к застряванию алгоритма в локальных минимумах.

Чтобы обойти эту проблему, Мошер и Лихи предложили использовать для
поиска активных токовых диполей алгоритм MUSIC~\ref{MUSIC, Schmidt}, разработанный
и использовавшийся ранее в радиопеленгации и сонарах.

Рассмотрим подробно суть метода в применении к данным ЭЭГ/МЭГ.
Начнем с рассмотрения порождающей модели сигнала на сенсорах, как мы это
уже делали для оценки фазовой синхронности (см.~\ref{gm_ts}).
На этот раз, однако, заложим в модель возможность свободной ориентации диполей.
От каждого активного токового диполя на коре будем иметь вклад на сенсорах вида:

\begin{equation}
    \mathbf{x}_k(t) =
        \begin{bmatrix}
            |                 & |              & |              \\
            \mathbf{g}_k^1    & \mathbf{g}_k^2 & \mathbf{g}_k^3 \\
            |                 & |              & |
        \end{bmatrix}
        \left(\begin{array}{ccc}
                s_{k,1}(t)\\
                s_{k,2}(t)\\
                s_{k,3}(t)
            \end{array}
        \right)
        % \mathbf{s}_{\xi}(t),
\end{equation},
где $k$"--- индекс токового диполя $s_{k,i}$"--- компоненты соответствующего
дипольного момента,
$\mathbf{g}_k^i$"--- вектора-топографии $k$-го токового диполя
для трех ориентаций тока ($i=1,2,3$).
Тогда вклад от всех активных токовых диполей будет виден на сенсорах как

\begin{equation}
    \mathbf{x}(t) = \mathbf{G} \mathbf{s}(t) + \mathbf{\omega}(t),
    \label{gm_music}
\end{equation}

Здесь вновь $\mathbf{s}(t)$ --- $3n$-мерный вектор-столбец активаций источников,
$\mathbf{x}(t)$ --- $m$-мерный вектор-столбец сигналов на сенсорах,
$t$ --- время, а $\mathbf{G}$ --- $m \times 3n$ матрица линейного
отображения пространства источников пространство сенсоров.

Уравнение \ref{gm_music} задает соответствие между пространством источников и пространством
сенсоров для каждого временного среза $t$.
При условии, что было записано $T$ таких срезов,
можем переписать уравнение~\ref{gm_music} в матричной форме:

\begin{equation}
    \mathbf{X} = \mathbf{G} \mathbf{S} + \mathbf{\Omega}
    \label{gm_music_matrix}
\end{equation}

Заметим, что столбцы матрицы $\mathbf{X}$ порождаются линейными комбинациями
векторов-топографий активных токовых диполей, а значит все возможные конфигурации
наблюдаемого сигнала живут внутри некоторого линейного подпространства линейной оболочки этих
топографий. Это линейное подпространство называется \emph{подпространством сигнала}.
При этом количество активных токовых диполей $r$ задает размерность этого подпространства
Чтобы выделить его, применим к матрице $\mathbf{X}$
сингулярное разложение и зафиксируем первые $r$ левых собственных векторов:

\begin{gather}
    \mathbf{U}, \mathbf{S}, \mathbf{V} = svd(\mathbf{X}) \\
    \mathbf{U} =
        \begin{bmatrix}
            |                 &               & |            \\
            \mathbf{u}_1      & \cdots        & \mathbf{u}_m \\
            |                 &               & |
        \end{bmatrix}\\
    \mathbf{U}_r = 
        \begin{bmatrix}
            |                 &               & |            \\
            \mathbf{u}_1      & \cdots        & \mathbf{u}_r \\
            |                 &               & |
        \end{bmatrix}
    % \mathbf{U}_m = U[:,:m]
\end{gather}

Матрица $\mathbf{U}_r$ называется \emph{матрицей подпространства сигнала};
ее столбцы являются ортонормальным базисом этого подпространства.
Чтобы найти все активные токовые диполи, для каждого источника на коре
вычислают \emph{корреляцию подпространств} между подпространством сигнала и
линейной оболочкой трех топографий, соответствующих данному источнику.
Корреляция подпространств вычисляется как максимальное собственное число
произведения матрицы левых собственных векторов для топографий $k$-го источника
и матрицы подпространства сигнала:

\begin{gather}
    \mathbf{U}_g_k, \mathbf{S}_{k,g}, \mathbf{V}_{k,g} = svd\left(
            \begin{bmatrix}
                |                 & |              & |              \\
                \mathbf{g}_k^1    & \mathbf{g}_k^2 & \mathbf{g}_k^3 \\
                |                 & |              & |
            \end{bmatrix}
     \right)
\end{gather}


