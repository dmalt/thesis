%% Согласно ГОСТ Р 7.0.11-2011:
%% 5.3.3 В заключении диссертации излагают итоги выполненного исследования, рекомендации, перспективы дальнейшей разработки темы.
%% 9.2.3 В заключении автореферата диссертации излагают итоги данного исследования, рекомендации и перспективы дальнейшей разработки темы.

Мы описали новый метод обнаружения фазовой связности в выделенной полосе
частот, по неинвазивным данным МЭГ/ЭЭГ. Предложенный подход демонстрирует,
возможность выделять истинную линейную связь с нулевым сдвигом по фазе по
неинвазивными записям.  Это достигается за счет предложенной процедуры
проекции, работающей в пространстве сенсоров на кросс-спектральных матрицах и
позволяющей эффективно подавить вклад пространственной протечки сигнала в
действительную часть кросс-спектральной матрицы. Оказывается, что хотя
подпространство, модулируемое действительной частью исходного кросс-спектра и
подпространство, включающее мощность протечки сигнала перекрываются, достаточно
просто построить пространственный проектор, чтобы подавить большую часть
протечки сигнала.  (например, 95, см. рис. 3), и при этом сохранить
чувствительность к большинству источников с нулевой фазовой связью. Как
показано на примере численного моделирования, предлагаемая методика позволяет
сохранить чувствительность для всего диапазона значений фазового сдвига, см. рис. 9.

Используя реалистичное моделирование, мы исследовали предлагаемую методику
(PSIICOS) и сравненили ее эффективность с набором других методов, таких как
DICS, iDICS, а также методом геометрической коррекции (GCS). Метод PSIICOS
стабильно превосходил три референтных метода в обеих смоделированных конфигурациях:
как с тремя сетями с перекрывающимися профилями активности и фиксированными положениями
узлов, так и в случае симуляций Монте-Карло на широком диапазоне реалистичных
условий зашумленности, а также значений фазового сдвига.  Важно отметить, что
метод PSIICOS показал равномерное качество решений для всего диапазона средних
значений фазового сдвига.

В последние годы появилось большое количество методов обнаружения
функциональной связности.  Если бы мы имели доступ к истинным сигналам на коре,
отражающим активность каждого отдельного узла сети, то можно было бы
использовать функцию когерентности, отражающую линейную (с точки зрения теории
линейных стационарных систем) связь между сигналами.  Однако активность
корковых генераторов, измеренная неинвазивно, доступна только в виде смеси
сигналов активации из нескольких источников и, таким образом, прямое
использование сигнала сенсоров приводят к ошибочным результатам: эффект
протечки сигнала маскирует истинную функциональную связь. Для решения этой
проблемы (Nolte et al.(2004a)) предложил использовать мнимую часть
кросс-спектра в качестве статистики на уровне сенсоров, которая не чувствительна к
протечке сигнала.

Это вызвало появление ряда методов, например (Stam et al. (2007), Vinck et al.
(2011), Эвальд и др. (2014)), использующих мнимую часть кросс-спектра на уровне
сенсоров.  Хотя некоторые из них и дают преимущество перед методом imCoh, все
они не в способны обнаруживать сети с нулевой фазовой задержкой, так как мнимая
часть когеренции функционально независима от действительной части исходного
кросс-спектра в пространстве источников, который несет информацию об истинной
связности с нулевой разностью фаз. Для нулевых или близких к нулю средних
фазовых сдвигов ОСШ мнимой части кросс-спектра на уровне сенсоров недостаточно
для надежной детекции, см. рис. 9. В то же время, использование необработанной
действительной части кросс-спектра не представляется возможным из-за эффекта
пространственной протечки. Как мы показали в данной работе, использование
обратного оператора на базе LCMV для разделения данных сенсоров, как это
предлагается в методике DICS, не обеспечивает необходимой точности, а
последующее использование когерентности в пространстве
источников не позволяет получать решения удовлетворительного качества при
реалистичных значениях ОСШ.

Представленный здесь подход наиболее тесно связан со схемой геометрической
коррекции (GCS, Wens et al., 2015), первоначально введенной в применение для
анализа связи между огибающими сигналов в пространстве источников в качестве
альтернативы методам ортогонализации на основе временной структуры активации,
(Hipp et al., 2012), (Colclough et al., 2015). Подход GCS предполагает
использование топографии фиксированного узла в сочетании с фильтром на основе
обратного оператора для некоторого другого источника с которым меряется
связность фиксированного узла. Метод GCS устраняет эффект пространственной
протечки, связанный только с этим фиксированным источником.  В сравнении с этим
вместо использования топографии фиксированного источника и устранения эффекта
протечки сигнала только от него, подход PSIICOS работает в пространстве внешних
произведений топографий пар источников и предполагает создание проектора,
который учитывает вклад в протечку сигнала от всех возможных источников.
Использование сингулярного разложения позволяет построить эффективный оператор
проекции, позволяющий сконцентрировать наибольшее количество нежелательной
мощности протечки сигнала в подпространстве фиксированного ранга. Эта операция
проецирования применяется к матрице кросс-спектра в пространстве сенсоров.

Подобно GCS, PSIICOS позволяет визуализировать динамику отражающего картину
взаимодействий кросс-спектра в пространстве сенсоров и позволяет проводить анализ
в том числе не переходя в пространство источников (применение к анализу на уровне сенсоров
в данной работе не рассматривается). Например,
учитывая растущую значимость методов машинного обучения в анализе
нейрофизиологических данных, операция проекции, составляющая основу PSIICOS,
позволяет получать признаки, отражающие истинную связность в
относительно компактном пространстве сигналов на сенсорах. Проекция PSIICOS также
может быть применена к отдельным векторизованным внешним произведениям
данных на уровне сенсоров, а затем использована для обнаружения корковых участков со
значимыми корреляциями огибающих.

Как мы показали, подход с использованием порождающего уравнения позволяет
интерпретировать задачу оценки параметров порождающей модели кросс-спектра (c
ij) как стандартную недоопределенную проблему линейной регрессии,
распространенную в неинвазивных методах нейровизуализации.  Такой подход
позволяет получить четкий способ введения столь необходимой априорной
информации в задачу оценки связности. Эта информация может быть получена с
помощью диффузионной тензорной визуализации и представлена с использованием
вероятностного распределения, которое затем естественным образом используется в
рамках Байесовской парадигмы.
Менее специфичные, упрощенные априорные распределения, основанные на
пространственной разреженности, также могут быть использованы, и подход,
аналогичный описанному в (Strohmeier et al., 2016), основанный на смешанных
дробных нормах, может быть использован для построения
разреженных решателей, объясняющих наблюдаемый пространственный кросс-спектр на
уровне сенсоров активностью небольшого числа элементарных сетей в пространстве
источников.

Также, следуя по пути параметрического оценивания, можно PSIICOS позволяет применять
обобщения методов подгонки диполя, включая модифицированный алгоритм RAP-MUSIC, примененный к
матрице кросс-спектра с удаленной протечкой сигнала. Фактически, (Ewald et al. (2014)) описывает
использование MUSIC-подхода для анализа мнимой части кросс-спектра на
основе MUSIC-метрик, но, как уже отмечалось ранее, из-за использования только
мнимой части кросс-спектра, предлагаемый подход не чувствителен к сетям с
нулевыми фазовыми задержками. Кроме того, время вычисления для метода, описанного в
(Ewald et al. (2014)), велико, и авторы прибегают к двухэтапной процедуре,
чтобы избежать сканирования по $N^2$ парам источников.
Векторизованная форма кросс-спектра и соответствующее порождающее уравнение могут послужить основой
для разработки подхода RAP-MUSIC, при котором элементарные сети заменяют диполи
в исходных выкладках для этой методики (Мошер и Лихи (1999)). RAP-MUSIC
предполагает построение рекурсивных проекций от подпространства, образуемого
топографиями источников, обнаруженных на предыдущих итерациях. При применении к
векторизованному кросс-спектру для удаления обнаруженной элементарной сети,
состоящей из узлов А и В, такая проекция удалит только вклад конкретной
пары, а спроецированный кросс-спектр сохранит остальные сети, образованные источником А,
со всеми остальными источниками, за исключением узла В, а также сети, образованные источником В,
со всеми остальными источниками кроме А. Это означает, что при применении
подхода RAP-MUSIC к векторизованному кросс-спектру с удаленной протечкой сигнала мы можем
получить возможность исследовать сложные сети, состоящие более чем из одной
пары узлов. Отдельного изучения требует вопрос, решает ли эта процедура проблему, поднятую в работе
Mahjoory et al. (2017), что оценки, полученные методом бимформинга, и глобальные решения MNE
приводят к разным картинам распределения связности.

В текущей векторизованной реализации в среде MATLAB расчет проекционной матрицы
для модели пространства источников с 7000 узлами занимает менее одной секунды
расчетного времени и требует вычисления лишь одинажды для каждого испытуемого,
если предположить, что положения сенсоров фиксированы или что в данных была
сделана поправка на движения испытуемого (такая поправка возможна
использованием метода tSSS).  Векторизованная реализация сканирования по
пространству источников размером 7000 на 7000 узлов занимает полсекунды
вычислительного времени на современном ноутбуке.  Таким образом предложенный
подход является вычислительно эффективным и делает возможным проведение анализа
на основе симуляций Монте-Карло для исследования устойчивости наблюдаемых
сетей, аналогичного проведенному в данной работе.

Современные метрики взаимодействия в пространстве источников для MEG/EEG,
игнорирующие связь с нулевым сдвигом по фазе, приводят как к ложноположительным
Matias Palva и др. (2018), так и к ложноотрицательным срабатываниям. Хотя первая проблема была
решена недавно (например, Wang et al. (2018)), решение второй проблемы до сих
пор не было предложено. В этой работе мы приводим первую демонстрацию
возможности неинвазивного картирования истинной мгновенной линейной связности.
Учитывая убедительные доказательства существования такой малолатентной связи,
как это видно при инвазивных исследованиях на животных, мы полагаем, что
описанная здесь методика PSIICOS значительно расширит возможности современных
функциональных инструментов сетевого анализа.

% Strengths and limitations of PSIICOS framework

Предложенный здесь подход представляет собой новое решение для изучения
взаимодействий в данных МЭГ и может выборочно решать проблему пространственной
протечки даже в случае нулевой или близкой к нулю фазовой связности. Это
позволяет преодолеть ограничения, присущие методам, использующим мнимую часть
кросс-спектра или основанные на временной структуре данных ортогонализационные
подходы, которые по определению игнорируют нулевые фазовые взаимодействия (и
имеют слабую чувствительность к околонулевым фазовым сдвигам). Отметим, однако,
что PSIICOS требует статистической процедуры, основанной на бутстраппинге, и
что он не полностью застрахован от ложноположительных срабатываний при наличии
несвязанных источников с профилями мощности, которые значительно выше, чем у
взаимодействующих источников.

Основное внимание в данной работе мы уделяем новой проекционной схеме,
позволяющей существенно подавить вклад пространственной протечки в кросс-спектр на уровне сенсоров
и получить новое порождающее уравнение (9), позволяющее представить задачу
оценки фазовой связности как задачу оценки источника, но в пространстве пар
сигналов сенсоров. Для проведения необходимой валидации метода мы
выбрали максимально простую стратегию поиска источников для этого уравнения.
Даже при таком простом подходе к оценке наши результаты демонстрируют потенциально
более высокую производительность предлагаемой методики по сравнению с рядом
других релевантных методов и почти одинаковую чувствительность ко всему
диапазону средних значений разности фаз между временными рядами связанных
источников, включая нулевую и близкую к нулю фазовые задержки. Основываясь на
работе, описанной в работе Дарваса F (2005), мы также предложили процедуру
бутстрапа, которая может быть использована для проверки стабильности
наблюдаемого результата.

В случае наличия в данных истинной связности предлагаемая процедура бутстрапа 
имеет низкую вероятность генерации сети из пары активных, но функционально
не связанных источников, до тех пор, пока мощность этих источников
не будет существенно превышать мощность в узлах истинных сетей.
Если данные содержат несколько не связанных между собой функционально, но обладающих высокой мощностью
источников, предлагаемая процедура может привести к ложным срабатываниям. Кроме
того, учитывая описанный способ отбора сетей по верхним значениям метрики
сканирования $\rho$, мы, скорее всего, упустим некоторые истинные сети.

Лучшим способом решения обеих этих проблем является разработка эффективного
статистического теста, работающего на основе распределения для нулевой гипотезы.
Однако чтобы быть полезным, этот тест должен сохранять распределение мощности
в пространстве сенсоров, разрушая при этом нулевую и близкую к нулю
фазовую связность. Тесты, разработанные до сих пор для оценки линейной
синхронизации, в основном адаптированы к мерам, не чувствительным к мгновенной
связности. Более того, применение методов, основанных на рандомизации временных
рядов компонент ICA, не обеспечивает подходящего решения, когда необходимо
обнаружить связность с околонулевой фазовой задержкой. Для решения этих проблем необходим
подход, основанный на фазовой рандомизации источников,
которая бы уничтожала взаимные фазы, но сохраняла бы плавность фазового ответа
отдельных активаций. Однако, поскольку алгоритмы генерации суррогатных данных
соответствуют свойствам исходных данных в пространстве сенсоров (Хауфе и Эвальд 2016 г.),
то пользуясь такими методами может быть трудно отличить мгновенную корреляцию,
вызванную исключительно объемной проводимостью, от истинной связи с нулевой разностью фаз.
Поэтому необходимы более консолидированные усилия для создания надежной системы
статистического тестирования, адаптированной к методам анализа связи, таким как
описанная здесь.

Несмотря на эти ограничения, PSIICOS представляет собой первую попытку
обнаружить по данным МЭГ взаимодействие с нулевой и близкой к нулю фазовой
задержкой.  Насколько нам известно, задача оценки линейного взаимодействия на
основе MEG/EEG впервые представлена как многомерная регрессионная задача,
аналогичная той, которая встречается в классической обратной задаче для этого
типа данных. Такое рассмотрение открывает богатый спектр возможностей для
адаптации множества регуляризационных или параметрических методик,
разработанных в этой области, для решения проблемы оценки функциональной связи.

\begin{enumerate}
  \item Был проведен обзор исследований изменения функциональной коннективности
      мозга при различных паталогиях.
  \item Был разработан метод очистки данных ЭЭГ и МЭГ от протечки сигнала, на
      основе которого было разработано семейство алгоритмов оценки фазовой синхронности,
      позволяющих находить сети с близкими к нулю фазовыми задержками.
  \item Задача оценки фазовой синхронности в условиях протечки сигнала была
      сформулирована и решена как задача оптимальной фильтрации.
  \item Был предложен алгоритм, позволяющий обнаруживать сети с близкими
      к нулю фазовыми задержками, оптимальный в глобальном смысле и позволяющий
      справиться с проблемой ложноположительных срабатываний второго рода, вызванных
      протечкой сигнала.
  \item Было проведено численное исследование свойств предложенной
      проекции, показавшее, что разработанная методика позволяет
      подавить вклад подпространства протечки сигнала в оцененную
      на сенсорах матрицу кросс-спектральной плотности мощности.
  \item Численное исследование свойств метода проекции показало, что
      разработанная методика позволяет находить сети с близкими к нулю фазовыми задержками в условиях
      неинвазивных МЭГ измерений, которые характеризуются значительной протечкой
      сигнала между источниками.
  \item Сравнение с имеющимися на данный момент алгоритмами оценки коннективности
      по неинвазивным данным на основе симуляций показало значительное превосходство
      предложенной техники обнаружения сетей в условиях малых фазовых задержек.
  \item Применение метода очистки от протечки сигнала к реальным данным позволило
      обнаружить физиологически правдоподобные сети, которые невозможно обнаружить
      другими способами.
  \item Было проведенно численное исследование влияния значений ранга предложенной проекции
      на свойства алгоритма, которое позволило получить эвристику для выбора ранга.
  \item Численное исследование влияния неточностей прямой модели на качество решений
      предложенного алгоритма показало, что характерные для реальных записей
      диапазоны ошибок в оценке прямой модели слабо сказываются на качестве
      получаемых решений.
  \item Для выполнения поставленных задач был создан
      пакет утилит в среде MATLAB, в который входят средства генерации тестовых
      данных, визуализации пространственной и временной структуры сетей и наконец
      программные реализации разработанных и использованных для валидации алгоритмов.
  \item Наработки, полученные в ходе работы над данной диссертацией,
      были внедрены в пакеты программ Visbrain и Neuropycon, доступные для публичного
      использования.
\end{enumerate}
