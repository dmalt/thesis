\chapter{Анализ свойств метода оценки синхронизации с нулевым фазовым сдвигом.} \label{chapt3}
В этой главе мы исследуем свойства методов, основанных на проекции от протечки сигнала,
а также сравним результаты их работы с методами, являющимися на данный
момент стандартом в области исследований функциональной коннективности.

\section{Исследование воспроизводимости решения для метода PSIICOS}.
Для проверки стабильности решений мы использовали процедуру бутстрэпа, аналогичную
описанной в \cite{Darvas_2005}.
Сначала мы генерируем $B$ различных кросс-спектральных временных ряда (CT), полученных
усредненем по эпохам, индексы которых берутся как сочетания с повторениями, набранные
с равными вероятностями из полного набора индексов.

Затем на каждой итерации бутстрэпа небольшое
число пар $m$, соответствующих сетям с наибольшими
значениями кросс-спектральных коэффициентов, полученными в результате применения
алгоритма PSIICOS, группируются в несколько кластеров в соответствии с процедурой
попарной пространственной кластеризации (Pairwise Spatial Clustering, \cite{Zalesky_2012}).

Для каждого полученного кластера мы рассчитываем среднюю сеть. Для этого мы усредняем
координаты начал и концов сетей в кластере, предварительно ориентируя их таким
образом, чтобы концы были направлены в одну полуплоскость.
Средние сети, полученных на каждом шаге бутстрэпа, сохраняются.

Для количественной оценки пространственного разброса полученных таким образом средних сетей
мы определяем расстояние между парой сетей как минимум по двум возможным ориентациям
сетей от суммы евклидовых расстояний между их концами.

В соответствии с определенной таким образом функцией расстояния мы определяем индекс
воспроизводимости $\eta$ как единицу, деленную на среднее по всем $B$ средним сетям расстояние
от сети до ближайшего соседа.

\section{Симуляции методом Монте-Карло.}
Чтобы сравнить предложенный алгоритм с другими методами, оценивающими коннективность
в пространстве источников, мы сконструировали ряд реалистичных симуляций.

Для симуляций использовали поверхность коры реального испытуемого,
реконструированную по МРТ-снимкам при помощи пакета программ Freesurfer.
Для аппроксимации этой поверхности трехмерной вычислительной сеткой мы
использовали 15000 узлов. Для этой поверхности мы рассчитали матрицу прямой модели
высокого разрешения $\matr{G}^{HR}$, в которой на каждую точку коры приходится
две топографии, соответствующие модели со свободной ориентацией диполя для МЭГ
в случае сферической симметрии. Эти топографии мы рассчитывали при помощи метода главных
компонент для $M \times 3$ прямой модели для отдельно взятой точки, выкидывая
компоненту с наименьшим собственным числом.

Мы симулировали 100 повторений (эпох) эксперимента, в котором за стимулом следовал
всплеск индуцированной активности на частоте 10 Гц. Индуцированная активность
симулировалась как пары синусоид, не привязанные по фазе к стимулу, но с неслучайной
разностью фаз по отношению друг к другу. Их разность фаз $\delta\phi$ выбиралась
для каждой эпохи случайно из равномерного распределения на отрезке $[-\pi/4, \pi/4]$.

Шум мозга мы моделировали как $Q=1000$ кортикальных источников, активность которых
никак не связана со стимулом. Положения на коре этих источников
выбирались для каждой эпохи независимо и случайно.

Для моделирования их временных профилей активации мы фильтровали реализации гауссовского
случайного процесса полосовым БИХ-фильтром пятого порядка в полосах частот,
соответствующих тета (4--7 Гц), альфа (8--12 Гц), бета (15--30 Гц) и гамма (30--50, 50--70 Гц)
активности. Относительные вклады этих полос частот мы взвешивали таким образом, чтобы
итоговая спектральная плотность мощности имела вид $1/f$, соответствующий реалистичному
спектру данных МЭГ. Далее общий вклад шума мозга в итоговый сигнал взвешивался,
чтобы соответствовать типичным соотношениям сигнал-шум, наблюдаемым в реальных
записях.

Для отображения шумовых источнков на сенсоры умножали каждый срез полученных временных
профилей на соответствующие столбцы матрицы $\matr{G}^{HR}$ и складывали результаты.

ОСШ в симуляционных данных мы определяли как отношение фробениусовских
норм матрицы данных и матрицы шума, отфильтрованных в целевой полосе (8--12 Гц).

Для вычисления матрицы кросс-спектральной плотности на сенсорах, соответствующей целевой полосе,
мы проводили следующую процедуру.
Мы фильтровали симулированный сигнал на каждом сенсоре в диапазоне (8--12 Гц) и
вычисляли аналитический сигнал.
Затем для каждого временного среза мы брали внешние произведения с сопряжением
комплекснозначного вектора аналитических сигналов на сенсорах и усредняли их по эпохам.

Чтобы наши симуляции были больше приближены к реальным условиям, когда настоящие источники
не всегда точно ложатся на узлы вычислительной сетки, мы использовали сетку с высоким
разрешением (15000 узлов) только для симуляции данных. Для оценки
коннективности по симуляционным данным мы использовали в 10 раз более разреженную сетку,
насчитывающую 1503 узла.

В качестве метрик качества мы использовали кривые Receiver Operating Characteristics (ROC)
и Precision-Recall (PR). Метрика Precision-Recall лучше работает в ситуации, когда
число истинно положительных мало при большом количестве возможностей выбора.
ROC-кривая, или график \emph{чувствительность~--- 1 - специфчность} в этом случае информативна
только для очень высоких значений специфичности.

Для каждой реализации метода Монте-Карло мы считали ROC-кривую и затем
усредняли кривые по ансамблю из 1000 реализаций. Сравнения проводились для двух различных
значений ОСШ: 1 и 0.2. 

Так как целевой характеристикой предложенного метода является одинаковое качество
решений для произвольных фазовых задержек между связанными источниками, мы сравнивали
поведение методов для двух различных фазовых задержек $\delta\phi=\pi/2-\pi/20$ и
$\delta\phi=\pi/20$ радиан. Также мы исследовали поведение нашего метода на
равномерной сетке значений фазового сдвига.




\clearpage
