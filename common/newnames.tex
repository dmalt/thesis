% Новые переменные, которые могут использоваться во всём проекте
% ГОСТ 7.0.11-2011
% 9.2 Оформление текста автореферата диссертации
% 9.2.1 Общая характеристика работы включает в себя следующие основные структурные
% элементы:
% актуальность темы исследования;
\newcommand{\actualityTXT}{Актуальность темы.}
\newcommand{\actualityTXTEng}{Actuality of the work.}
% степень ее разработанности;
\newcommand{\progressTXT}{Степень разработанности темы.}
\newcommand{\progressTXTEng}{Степень разработанности темы.}
% цели и задачи;
\newcommand{\aimTXT}{Целью}
\newcommand{\aimTXTEng}{The goal}

\newcommand{\tasksTXT}{задачи}
\newcommand{\tasksTXTEng}{tasks}
% научную новизну;
\newcommand{\noveltyTXT}{Научная новизна:}
\newcommand{\noveltyTXTEng}{The novelty}
% теоретическую и практическую значимость работы;
%\newcommand{\influenceTXT}{Теоретическая и практическая значимость}
% или чаще используют просто
\newcommand{\influenceTXT}{Теоретическая и практическая значимость.}
\newcommand{\influenceTXTEng}{Theoretical and practical significance.}
% методологию и методы исследования;
\newcommand{\methodsTXT}{Mетодология и методы исследования.}
\newcommand{\methodsTXTEng}{Methodology and research methods.}
% положения, выносимые на защиту;
\newcommand{\defpositionsTXT}{Основные положения, выносимые на~защиту:}
\newcommand{\defpositionsTXTEng}{Main provisions for the defence:}
% степень достоверности и апробацию результатов.
\newcommand{\reliabilityTXT}{Достоверность}
\newcommand{\reliabilityTXTEng}{The reliability}

\newcommand{\probationTXT}{Апробация работы.}
\newcommand{\probationTXTEng}{Publications and probation of the work.}

\newcommand{\contributionTXT}{Личный вклад.}
\newcommand{\contributionTXTEng}{Personal contribution into the main provisions for the defense.}

\newcommand{\publicationsTXT}{Публикации.}
\newcommand{\publicationsTXTEng}{Publications.}


\newcommand{\authorbibtitle}{Публикации автора по теме диссертации}
\newcommand{\authorbibtitleEng}{Author's publications on the topic of the dissertation study}
\newcommand{\vakbibtitle}{В изданиях из списка ВАК РФ}
\newcommand{\notvakbibtitle}{В прочих изданиях}
\newcommand{\confbibtitle}{В сборниках трудов конференций}
\newcommand{\fullbibtitle}{Список литературы} % (ГОСТ Р 7.0.11-2011, 4)
\newcommand{\fullbibtitleEng}{Bibliography} % (ГОСТ Р 7.0.11-2011, 4)

\newcommand{\Cp}[1]{\mathbf{C}^{\mathbf{#1 #1}}} % кросс-спектр
\newcommand{\Tf}[1]{\hat{\mathbf{#1}}} %
\newcommand*\conj[1]{#1^*}
\newcommand*\Expect[1]{\mathbf{E}\Big\{#1\Big\}}
% \newcommand*\abs[1]{\lvert#1\rvert}

\DeclarePairedDelimiter\abs{\lvert}{\rvert}%
\makeatletter
\let\oldabs\abs
\def\abs{\@ifstar{\oldabs}{\oldabs*}}
\makeatother

% \newcommand*\def[1]{\stackrel{def}{#1}}
\newcommand\defeq{\mathrel{\overset{\makebox[0pt]{\mbox{\normalfont\tiny\sffamily def}}}{=}}}
% \newcommand*\Laplace{\mathop{}\!\mathbin\bigtriangleup}
\newcommand*\Laplace{\Delta}

\newcommand{\norm}[1]{\left\lVert#1\right\rVert}
\DeclarePairedDelimiter\set\{\}
\DeclareMathOperator{\tr}{tr}
\DeclareMathOperator{\coh}{coh}
\DeclareMathOperator{\sign}{sign}
\DeclareMathOperator*{\argmin}{arg\,min}
\DeclareMathOperator{\prox}{prox}
\DeclareMathOperator{\diag}{diag}
\DeclareMathOperator*{\argmax}{arg\,max}
\DeclareMathOperator*{\trace}{Tr}

\def\D{\mathrm{d}}  % differential

% matrices and vectors typesetting
\newcommand{\mG}{\mathbf{G}} % matrix G
\newcommand{\vs}{\mathbf{s}} % vector s
\newcommand{\vx}{\mathbf{x}} % vector x
\newcommand{\vomega}{\boldsymbol{\omega}} % vector omega

\renewcommand{\vec}[1]{\mathbf{#1}}
\newcommand{\matr}[1]{\mathbf{#1}}
